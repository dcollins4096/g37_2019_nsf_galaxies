\documentclass[11pt]{article}
\usepackage{fullpage}
\usepackage{subfigure}
\usepackage{graphics}
\usepackage{epsfig}
\usepackage{graphicx}
\usepackage{natbib}
\usepackage{xspace}
\usepackage{amsmath}
\usepackage{amsfonts}
\usepackage{amssymb}
\usepackage{wrapfig}
\usepackage{multirow}
%\usepackage{psfig}
\usepackage{rotating}

%\setlength{\topmargin}{+0.1in}
%\setlength{\oddsidemargin}{-0.125in}
%\setlength{\evensidemargin}{-0.125in}
\setlength{\textheight}{9.0in}
\setlength{\textwidth}{6.5in}
%\bibliographystyle{apj}
\newcommand{\apj}{ApJ}
\newcommand{\aj}{AJ}
\newcommand{\apjl}{ApJL}
\newcommand{\mnras}{MNRAS}
\newcommand{\apjs}{ApJS}
\newcommand{\pasp}{PASP}
\newcommand{\araa}{ARA\&A}
\newcommand{\aap}{A\&A}
\newcommand{\aaps}{A\&AS}
\newcommand{\pasj}{PASJ}
\newcommand{\prd}{Phys. Rev. D}
\newcommand{\nat}{Nature}
\newcommand{\physrep}{Physics Reports}
\def\enzo{{\sc Enzo}}

%\RequirePackage{natbib}
\pagestyle{empty}

\begin{document}

\vspace{-6mm}
\begin{center} 
\bfseries\uppercase{Section B -- Project Summary}
\end{center}
\vspace{-2mm}


\noindent \textbf{Overview:}
Magnetic fields in the Milky Way are present at all size scales, and impact its dynamics and observable properties.  Most notably magnetic fields impact the formation of stars, and fill the sky with polarized light.  Thus the behavior and evolution of this magnetic field is of great interest.

We will simulate how, when, and where magnetic fields are created in the Milky Way and similar sized galaxies.  We will use these simulations to produce high-resolution synthetic polarized dust and synchrotron maps for the study of foregrounds for the polarized cosmic microwave background (CMB). We will also study the correlations in the magnetic field as they impact the dynamics of the ISM in the galaxy.

Our simulations will proceed in 2 stages in order to capture the complete history of the magnetic field.  The first stage will simulate the formation of a Milky Way sized galaxy from cosmological initial conditions.  The second stage will extract that galaxy and simulate it in an isolated, high-resolution context.  This two-pronged approach allows us to capture most of the physical mechanisms responsible for magnetic field growth.   The cosmological simulations will capture the environmental processes and to some limited extend dynamo processes, while the isolated simulations will have resolution to better capture the dynamo.  By extracting galaxies and their local environments from the cosmological simulations we will accurately model the circumgalactic material. In addition to containing a substantial amount of mass and likely magnetic energy, the environment above and below is important for the functioning of an ordered dynamo.  

\textbf{Intellectual merit:} This project is novel because it will use high resolution magnetohydrodynamics simulations to self-consistently follow the evolution of plasma over a huge range of astrophysically important length, density, and temporal scales, thus bridging the gap between the cosmological structure formation that results in galactic-scale magnetic fields and the star-forming molecular clouds that form out of the magnetized interstellar medium.  This is crucial because an improved understanding of how the magnetized interstellar medium (ISM) forms within galaxies will lead to more accurate understanding of both observations and dynamics of the Milky Way and similar galaxies.


\textbf{Broader impacts:} 
Our proposed work will have significant impact on scientists in training, who will learn to use cutting-edge numerical tools will develop critical skills in scientific software development and data analysis.  The tools developed as part of this work will be incorporated into widely used open-source software projects and all simulation and analysis data products will be made publicly available, maximizing the return on this investment by enabling the community to more easily build upon this work.  This work, as well as a myriad of other physics and astronomy observations, will be shared with the public via monthly ``Ask A Scientist'' events in Tallahassee.
%%%%%%%%

\end{document}
