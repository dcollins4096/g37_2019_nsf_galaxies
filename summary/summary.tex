\documentclass[11pt]{article}
\usepackage{fullpage}
\usepackage{subfigure}
\usepackage{graphics}
\usepackage{epsfig}
\usepackage{graphicx}
\usepackage{natbib}
\usepackage{xspace}
\usepackage{amsmath}
\usepackage{amsfonts}
\usepackage{amssymb}
\usepackage{wrapfig}
\usepackage{multirow}
%\usepackage{psfig}
\usepackage{rotating}

%\setlength{\topmargin}{+0.1in}
%\setlength{\oddsidemargin}{-0.125in}
%\setlength{\evensidemargin}{-0.125in}
\setlength{\textheight}{9.0in}
\setlength{\textwidth}{6.5in}
%\bibliographystyle{apj}
\newcommand{\apj}{ApJ}
\newcommand{\aj}{AJ}
\newcommand{\apjl}{ApJL}
\newcommand{\mnras}{MNRAS}
\newcommand{\apjs}{ApJS}
\newcommand{\pasp}{PASP}
\newcommand{\araa}{ARA\&A}
\newcommand{\aap}{A\&A}
\newcommand{\aaps}{A\&AS}
\newcommand{\pasj}{PASJ}
\newcommand{\prd}{Phys. Rev. D}
\newcommand{\nat}{Nature}
\newcommand{\physrep}{Physics Reports}
\def\enzo{{\sc Enzo}}

%\RequirePackage{natbib}
\pagestyle{empty}

\begin{document}

\vspace{-6mm}
\begin{center} 
\bfseries\uppercase{Section B -- Project Summary}
\end{center}
\vspace{-2mm}


\noindent \textbf{Overview:}
Observations show that magnetic fields are ubiquitous in the universe.
They can be seen in intergalactic space, and organized, dynamically
important magnetic fields thread the interstellar medium in galaxies
across cosmic time and regardless of galaxy size or morphology.
Molecular clouds form out of this magnetized plasma, and measurements
show that these clouds have substantial magnetic fields as well, which
 may be critical to star formation,  one of
the most important unsolved problems in astrophysics.  While magnetic
fields have been observed in all of these astrophysical regimes, and
there is a clear sequence of events -- galaxies form molecular clouds,
which in turn are the sites of star formation -- the ways in which
magnetic fields tie galaxies to molecular clouds, and thus potentially
affect star formation and the initial stellar mass function, are
poorly understood theoretically.  \textbf{There is a clear need for a
detailed, self-consistent model of cosmological structure formation
that can trace the evolution of magnetized gas to the physical scales
relevant for star formation.}

Our objective in this project is to understand how galactic-scale
magnetic fields affect the formation and evolution of molecular
clouds, and in turn how magnetic fields ejected from stars forming in
these clouds are amplified and ordered within galaxies.  We will
pursue this goal through the use of magnetohydrodynamic (MHD)
cosmological simulations of galaxy formation, which we will use to
inform (and be informed by) idealized, high resolution MHD simulations
of isolated galaxies  that resolve the
formation of magnetized molecular clouds.  This will both determine the impact of
magnetic fields on the bulk properties of these clouds and model the way in which
 fields are returned from stars to the interstellar medium. 
 Throughout our theoretical exploration, we will be
motivated by, and compare our results to, observations of magnetic
fields in the Milky Way, in nearby galaxies and the intergalactic medium, and in the high redshift
universe.  The end result of this
project will be a deep understanding of the connection between
galaxies, molecular clouds, and the magnetic fields that permeate
these objects.

\textbf{Intellectual merit:} This project is novel because it will use
high resolution magnetohydrodynamics simulations to self-consistently
follow the evolution of plasma over a huge range of astrophysically
important length, density, and temporal scales, thus bridging the gap
between the cosmological structure formation that results in
galactic-scale magnetic fields and the star-forming molecular clouds
that form out of the magnetized interstellar medium.  This is crucial
because an improved understanding of how magnetized molecular clouds
form within galaxies will lead to more accurate initial conditions for
targeted studies of star formation, and will provide an opportunity to
model that critical process in a more realistic way.  The results of
this project will facilitate our interpretation of observations of
magnetic fields in the intergalactic medium, in both high redshift and
nearby galaxies, and in the Milky Way galaxy itself.

\textbf{Broader impacts:} Our proposed work will have significant
impact on scientists in training, who will learn to use cutting-edge
numerical tools will develop
critical skills in scientific software development and data analysis.
We will involve undergraduate and graduate students at  MSU and a
postdoctoral researcher at 
FSU in our research efforts, including
students from under-represented groups.  The tools developed as part
of this work will be incorporated into widely used open-source
software projects and all simulation and analysis data products will
be made publicly available, maximizing the return on this investment
by enabling the community to more easily build upon this work.
Scientific results from this program will be visualized by members of
our collaboration, and will be disseminated to the public via our
pre-existing collaborations with planetaria and museums, via the
Internet, and as part of outreach talks given by members of this
project.  The simulation data produced as a result of this project
will be used in courses at both
MSU and FSU, and the curricular materials will be made
available via the Internet.

%%%%%%%%

\end{document}
