\documentclass[11pt]{article}
\usepackage{fullpage}
\usepackage{subfigure}
\usepackage{graphics}
\usepackage{epsfig}
\usepackage{graphicx}
\usepackage{natbib}
\usepackage{xspace}
\usepackage{amsmath}
\usepackage{amsfonts}
\usepackage{amssymb}
\usepackage{wrapfig}
\usepackage{multirow}
\usepackage{color}
%\usepackage{psfig}
\usepackage{rotating}

%\setlength{\topmargin}{+0.1in}
%\setlength{\oddsidemargin}{-0.125in}
%\setlength{\evensidemargin}{-0.125in}
%\setlength{\textheight}{9.0in}
%\setlength{\textwidth}{6.5in}
%\bibliographystyle{apj}
\newcommand{\apj}{ApJ}
\newcommand{\aj}{AJ}
\newcommand{\apjl}{ApJL}
\newcommand{\mnras}{MNRAS}
\newcommand{\apjs}{ApJS}
\newcommand{\pasp}{PASP}
\newcommand{\araa}{ARA\&A}
\newcommand{\aap}{A\&A}
\newcommand{\aaps}{A\&AS}
\newcommand{\pasj}{PASJ}
\newcommand{\prd}{Phys. Rev. D}
\newcommand{\nat}{Nature}
\newcommand{\physrep}{Physics Reports}
\newcommand{\red}[1]{\textcolor{red}{#1}}

%\RequirePackage{natbib}
\pagestyle{empty}

\begin{document}

\vspace{-6mm}
\begin{center} 
\bfseries\uppercase{Section I -- Facilities, Equipment, and Other Resources }
\end{center}
%\vspace{-2mm}

At present, the PIs have no significant computing resources of their
own, beyond the standard desktop and laptop computing.

PI O'Shea and his students and postdoctoral researchers have access to
MSU's High Performance Computing (HPC) facility, which maintains six
clusters comprising a total of roughly 18,000 CPU cores, 200 GPU and
Intel Phi accelerators, and 60 TB of memory.  The theoretical peak
speed of the entire system is approximately 600 TFlops. The nodes are
connected via Infiniband, and share a high speed, 1.9 PB Lustre
filesystem.  There is currently no cost for computing on MSU HPCC
machines, which will be used for data analysis and running small
simulations.

Members of this collaboration have access to a range of NSF XSEDE
computing resources through the XSEDE Resource Allocation Committee
(XRAC) proposal system.  They have collectively been allocated
approximately one million node-hours for FY17 on various NSF
platforms (primarily Stampede2 at TACC and Comet at SDSC), and we
expect that future projects will also be supported by NSF
computational resources.  Furthermore, MSU (and PI O'Shea) are part of
the Great Lakes Consortium for Petascale Computing, which is
associated with the Blue Waters petascale supercomputer project.  PI
O’Shea is currently the recipient of an NSF Office of Cyber Infrastructure
Petascale Computing Resource Application (PRAC) grant, which ends on
March 31, 2018.

The Departments of Physics and Astronomy and Computational
Mathematics, Science and Engineering at Michigan State University
will provide use of their offices, departmental computers, associated
peripherals, and the help of computer and administrative support
staff.

\end{document}
