\documentclass[11pt]{article}
\usepackage{fullpage}
\usepackage{subfigure}
\usepackage{graphics}
\usepackage{epsfig}
\usepackage{graphicx}
\usepackage{natbib}
\usepackage{xspace}
\usepackage{amsmath}
\usepackage{amsfonts}
\usepackage{amssymb}
\usepackage{wrapfig}
\usepackage{multirow}
\usepackage{titlesec}
\usepackage{hyperref}

%\usepackage{psfig}
\usepackage{rotating}

%\setlength{\topmargin}{+0.1in}
%\setlength{\oddsidemargin}{-0.125in}
%\setlength{\evensidemargin}{-0.125in}
\setlength{\textheight}{9.0in}
\setlength{\textwidth}{6.5in}

%\pagestyle{empty}

\titleformat*{\section}{\large\bfseries}

%\pagestyle{empty}  % removes page numbers - DO THIS BEFORE SUBMISSION

\begin{document}


\vspace{-6mm}
\begin{center}
  \bfseries\uppercase{Section J -- postdoctoral mentoring plan}
\end{center}
\vspace{-2mm}


In order to achieve the demanding code development requirements of this project,
we will be hiring a postdoctoral researcher.
This scientist's  development will be enhanced not only by the research that will be
performed in conjunction with this project, but also a plan of structured
activities to ensure her professional development.  These are based on
guidelines from the National Academy of Science. These will include the
following items.

\medskip
\noindent An \textbf{Individual Development Plan} will be created based on
the postdoc's career goals both in the immediate future and long term.  
This
will be followed up yearly with formal written evaluation, and informal monthly
meetings to evaluate progress and, if need be, revise the plan.

\medskip
\noindent \textbf{Communication Skills} will be developed by regular
presentations of the work performed in conjunction with this proposal at both
local and national meetings.  Written communication will also be developed
through publication of the results of this research.  Travel will be assisted
through FSU's Office of
Postdoctoral Affairs which provides Postdoctoral Scholars Career Development Travel
Awards for conference and meeting travel.  

\medskip
\noindent \textbf{Community Code Development} skills will be developed through
regular interactions with the Enzo and yt communities.  This includes working
with distributed version control and source code conflict resolution, as well as
integrating and publishing software to the open source community.

\medskip
\noindent \textbf{Grant Writing} techniques will be developed through both
collaboration on NSF and NASA grants, as well as serving as PI on XSEDE computer
allocation grants, which allows postdoctoral researchers to serve as PI.

\medskip
\noindent \textbf{Networking Opportunities} will be developed through a variety
of channels, including:
collaborations with the University of Florida, where potential collaborations
have already begun; the anual Enzo developers' workshop, where the future
directions of the code project are discussed and major issues taken care of;
visiting astrophysics scholars that are leaders in the field; and finally travel
to national meetings on topics that are relevant to the postdoc's long term
career goals.

\medskip
\noindent \textbf{Mentoring Opportunities} will be fostered as the postdoc can
work with graduates and undergraduates on this and related projects.  



%Postdoctoral Researcher Mentoring Plan. Each proposal33 that requests funding to
%support postdoctoral researchers34 must include, as a supplementary document, a
%description of the mentoring activities that will be provided for such
%individuals. If a Postdoctoral Researcher Mentoring Plan is required, FastLane
%will not permit submission of a proposal if the Plan is missing. In no more than
%one page, the mentoring plan must describe the mentoring that will be provided
%to all postdoctoral researchers supported by the project, irrespective of
%whether they reside at the submitting organization, any subawardee organization,
%or at any organization participating in a simultaneously submitted collaborative
%project. Proposers are advised that the mentoring plan may not be used to
%circumvent the 15-page project description limitation. See GPG Chapter II.D.4
%for additional information on collaborative proposals. Mentoring activities
%provided to postdoctoral researchers supported on the project will be evaluated
%under the Broader Impacts review criterion.
%Examples of mentoring activities include, but are not limited to: career
%counseling; training in preparation of grant proposals, publications and
%presentations; guidance on ways to improve teaching and mentoring skills;
%guidance on how to effectively collaborate with researchers from diverse
%backgrounds and disciplinary areas; and training in responsible professional
%practices.

\end{document}
