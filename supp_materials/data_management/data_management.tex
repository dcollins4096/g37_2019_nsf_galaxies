\documentclass[11pt]{article}
\usepackage{fullpage}
\usepackage{subfigure}
\usepackage{graphics}
\usepackage{epsfig}
\usepackage{graphicx}
\usepackage{natbib}
\usepackage{xspace}
\usepackage{amsmath}
\usepackage{amsfonts}
\usepackage{amssymb}
\usepackage{wrapfig}
\usepackage{multirow}
\usepackage{titlesec}
\usepackage{hyperref}

%\usepackage{psfig}
\usepackage{rotating}

%\setlength{\topmargin}{+0.1in}
%\setlength{\oddsidemargin}{-0.125in}
%\setlength{\evensidemargin}{-0.125in}
\setlength{\textheight}{9.0in}
\setlength{\textwidth}{6.5in}
%\bibliographystyle{apj}
\newcommand{\apj}{ApJ}
\newcommand{\aj}{AJ}
\newcommand{\apjl}{ApJL}
\newcommand{\mnras}{MNRAS}
\newcommand{\apjs}{ApJS}
\newcommand{\pasp}{PASP}
\newcommand{\araa}{ARA\&A}
\newcommand{\aap}{A\&A}
\newcommand{\aaps}{A\&AS}
\newcommand{\pasj}{PASJ}
\newcommand{\prd}{Phys. Rev. D}
\newcommand{\nat}{Nature}
\newcommand{\physrep}{Physics Reports}
%\RequirePackage{natbib}
\pagestyle{empty}

\titleformat*{\section}{\large\bfseries}
\hypersetup{urlcolor=blue}
\def\enzo{Enzo}
\def\enzoe{Enzo-E}
\begin{document}

% NOTE: this document addresses
% http://www.nsf.gov/pubs/policydocs/pappguide/nsf11001/gpg_2.jsp#dmp

\vspace{-6mm}
\begin{center} 
\bfseries\uppercase{Section J -- data management plan}
\end{center}

\vspace{-3mm}
\section{Products of the research}
\vspace{-2mm}

The research described in this proposal will result in several types
of data.  The bulk of the data produced will be raw and processed
simulation data (typically in time series), with the latter including
individual fields extracted from the simulations, halo catalogs and
merger trees, and synthetic observations.
Additional types
of data produced will include scientific software and a website
describing the data products and software.  This data will be stored
exclusively in digital formats.
 
\vspace{-3mm}
\section{Data Formats}
\vspace{-2mm}

For simplicity, we break the discussion of data format down by the
data product:

\vspace{2mm}
\noindent \textbf{Raw simulation data} will be stored in large binary
files using the HDF5 data format.  This
is a commonly used scientific data format that has strong community
support.  The structure of the simulation datasets can be somewhat
complex, but is explained by in-file metadata and an accompanying text
 file that describes important information
about the grid patches.  These data outputs can be read by a variety
of publicly available, open source tools such as the YT data analysis
tool, as well as the widely used Paraview and
VisIt visualization tools.  If necessary, the YT tool can be used as
an intermediary to produce files that are more easily accessible to
other researchers.

\vspace{2mm}
\noindent \textbf{Processed simulation data}, including most analysis
products, will be stored in binary HDF5 files with easily
understandable file structures.  Images resulting from data processing will be stored in
standard file formats such as jpeg, tiff, or png, and any mock
observational data created will be stored using
the FITS file format.  All files, when appropriate, will have
accompanying metadata  to explain the contents of the file, including the
nature of it, units, and related information.  All processed data and
analysis outputs will be created by a suite of publicly available
tools.

\vspace{2mm}
\noindent \textbf{Scientific software:} Both \enzo\ and \enzoe\ are
publicly available.
Any additions made to either code as a part of this
project will be contributed back to the relevant projects.

\vspace{2mm}
\noindent \textbf{Website:} The website associated with this project
will be ASCII plaintext using html, php, javascript, and so on.

\vspace{-3mm}
\section{Access to Data and Data Sharing Practices and Policies}
\vspace{-3mm}

\noindent\textbf{Simulation data} will be of sufficient size (hundreds
of terabytes) that it will be difficult to make
this available via a website, and wildly impractical to download via
http.  As such, we will make it available to other scientists through
the National Data Service  and hosted by archives at
the national supercomputing centers (see part 5 of this document,
below) with no restrictions on its use.  Instructions on its access
will be given via the project website, and also in README files in the
top-level directories on the NDS site.  This will be done by the end
of the grant period, although if possible we will make data available
upon acceptance of the relevant peer-reviewed papers.

\noindent\textbf{Processed simulation data and analysis products} will
also be of large size (hundreds of gigabytes to terabytes), and it
will be impractical to make it all available via a website.  The most
bulky data will be made available via the National Data Service (see
directly above, and Part 5), and other products  will be made
available via a website.  This website will allow users to
download and use the data with no restrictions on its use.  This will
be done by the end of the grant period, although if possible we will
make data available upon acceptance of the relevant peer-reviewed
papers.

\noindent\textbf{Source code} will be contributed back to the \enzo\  or YT software
projects (as appropriate) by the completion of the grant period (and,
more likely, by the acceptance of the peer-reviewed papers using the
relevant source code).  This software will then be available without
restriction to any user who wishes to download, use, or modify it.

\noindent The \textbf{website} will be available on the Internet.  All materials,
including movies, images, and text, will be licensed for re-use with
appropriate credit given using a license such as the GNU Copyleft
license.

Given the nature of the work being done as a part of this project and
our collaboration's stance regarding open source scientific software
and reproducible research, we have no significant privacy,
confidentiality, or intellectual property requirements that pertain to
this research.

\vspace{-3mm}
\section{Policies for Re-Use, Re-Distribution and Production of
  Derivatives}
\vspace{-3mm}

All data created as a result of this project – including raw
simulation data, processed data, data analysis products, and source
code – will be freely available and usable by the public and by other
researchers.  The only condition we will \textbf{require} is citation
of our publications upon use of our data (raw, processed, or analysis
products).  We will further \textbf{request, but not require,} that
users of our software contribute any changes, improvements, or
additions that they make back to the appropriate open-source software
project.  Data or images that are copyrighted (by a journal, for
example) will be marked as such by watermark or text header in the
file.

\vspace{-3mm}
\section{Archiving of Data}
\vspace{-3mm}

Due to the type and quantity of the simulation data involved in this
project, data archiving is a significant concern.  Raw and processed
simulation data and data analysis products will be stored in the mass
storage facilities at the national supercomputing facilities
(primarily the National Center for Supercomputing Applications and the
Texas Advanced Computing Center) and accessible via the National Data
Service.  This data will be stored and replicated as long as the PI's
research group continues to be allocated computing time at these
institutes (which will be at least for the duration of this grant, and
likely significantly longer).  However, data archiving in perpetuity
cannot be guaranteed at present, due to the impermanent nature of the
data tapes used for storage.

Source code will be stored on github primarily.  While it is unclear what the future holds
for any individual provider, it seems likely that the Enzo and YT
projects will continue long into the future, and this source code will
propagate, along with these projects, to whatever Internet home the
projects find in the future.

\end{document}
