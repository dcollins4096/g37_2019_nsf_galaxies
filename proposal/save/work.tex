\vspace{-4mm}
\section{Proposed Work}
\vspace{-3mm}

In this project, we will use high dynamic range magnetohydrodynamical
(MHD) simulations to understand how galactic-scale magnetic fields
affect the formation and evolution of molecular clouds, and to
identify the characteristics of these magnetized clouds that pertain
to star formation.  Specifically, we will use a targeted series of
high resolution MHD simulations of galaxy formation, both in a
cosmological context (Section~\ref{sec:cosmo_sims}) and as isolated, idealized
galaxies (Section~\ref{sec:molcloud_sims}), to explore the evolution
of magnetic fields through cosmic time.  

Each series of simulations will use
three main magnetic evolution assumptions: \emph{dynamo-only} only simulations,
wherein a weak seed field fills space at the beginning of the simulation, and
two \emph{feedback} recipes, that vary the amount of field injected through
supernovae.  These \emph{feedback} recipes will use the toroidal injection
implemented in \enzo\ by \citet{Butsky17}.  This model deposits a fraction,
$\sigma$, of supernova energy in a toroidal configuration on the grid.  Our
large field run will clone \citep{Butsky17} and use
 $10^{43}$erg per $\msun$ of initial
stellar mass, and our low-field run will use $1\%$ of that value.  For each of
the observational diagnostics,
(Section~\ref{sec:observational_comps}) we will perform synthetic observations,
matching noise and telescope resolution of the respective instruments, and
perform Baysian estimation to compare the four populations.  We expect that the
\emph{dynamo-only} simulations will fall short, which will allow us to bracket
the viability of our simulations.  In addition, we will perform the analysis on
the unfiltered simulations, in the "perfect telescope" limit, which will allow
us to examine the ability of future SKA or Next Generation JVLA observations to
explore these regions.

These calculations will be run with the \enzo\ code
(Section~\ref{sec:codes}) using computing time secured through the
XSEDE and PRAC allocation processes (Section~\ref{sec:comptime_sims}), and at all
times will be guided by, and compared to, a range of observations of
magnetic fields (Section~\ref{sec:observational_comps}).  This will be
done through close collaboration between an expert in simulations of
cosmological structure formation (O'Shea), an expert in simulations of
molecular clouds and star formation (Collins), an expert in
observations of magnetic field structure (Mao), and two graduate
students, as described in Section~\ref{sec:timeline}.



% From the cosmological simulation 
% These simulations will be used as initial and boundary conditions 

% \red{This will feed back to cosmological simulations as improved star
% formationa nd feedback algorithms, etc.}

% \red{need to elaborate upon this a bit before we dive into the
%   simulation section: we'll do simulations (described in 3.1), compare
% to specific observations (3.2), and will do this in a way that is
% logically partitioned (3.3).}

% Section 3.1 simulations
\subsection{Simulating the Magnetized Universe}
\label{sec:simulations}
\vspace{-1mm}

\subsubsection{Cosmological simulations}
\label{sec:cosmo_sims}
\vspace{-2mm}

As described in Section~\ref{sec:motivation}, there is a great deal of
observational evidence that magnetic fields exist in galaxies over a
wide variety of masses and redshifts, and also in intergalactic space.
Furthermore, it appears that the magnetic fields in star-forming
galaxies build up relatively quickly, based on observations of $z
\simeq 2-3$ galaxies which show that the magnetic fields measured in
their interstellar medium (ISM) are roughly in equipartition with
other sources of ISM energy.  We will explore the evolution of
magnetic fields in the universe, both over time and in a wide range of
environments, by using a carefully-chosen series of cosmological
simulations.  At present, a collaboration of galaxy modelers
(led by PI O'Shea) is in the midst of running two sets of
large-volume cosmological simulations using the \enzo\ code -- one of
a 25 Mpc volume, and the other 75 Mpc, both with $1024^3$ root grid
cells and particles, and with multiple sets of physics in each set of
calculations and moderate physical resolution ($\Delta \mathrm{x} \simeq 400$~pc).  Taken together, these sets of calculations resolve the
formation of galaxies over three orders of magnitude in mass, from
$\simeq 10^{10}$ to $10^{13}$~M$_\odot$, which encompasses galaxies
ranging in size from the Small Magellenic Cloud through the largest
ellipticals.  These simulations will be completed by the time that the
grant period for this project begins, and we will mine these datasets
to find approximately a dozen galaxies whose masses, morphologies, and
formation histories probe the regimes of interest for understanding
magnetic field behavior, and which we will resimulate at high resolution.  We will include
the following galaxies in our list of follow-up simulations,
which will enable us to target essentially the entire range of
observations listed in Sections~\ref{sec:extragalactic}
and~\ref{sec:lowz}:

\begin{enumerate}

\item Four spiral galaxies of roughly Milky Way mass (M$_{gal} \simeq 1-2 \times
10^{12}$~M$_\odot$) that have a range of formation histories,
including galaxies that have rapid early merger rates as well as those
with smoother formation histories.  In particular, we will include at least one galaxy
with a history similar to the Milky Way, with no major
mergers after $z \sim 2$.  These will be compared to observations of
local spiral galaxies \cite[e.g.,][]{2014arXiv1411.1386V} and to the
magnetic fields in the ISM of our own Milky Way.

\item At least two galaxies that grow rapidly at early times and are
large enough in terms of stellar mass that they can be directly compared to observations of magnetic fields
in L$_*$ galaxies at $z \simeq 2-3$
\cite[e.g.,][]{2008Natur.454..302B,2008ApJ...676...70K,1998A&A...329..809A}.

\item Four dwarf galaxies of masses $\simeq 10^{10}$ and $10^{11}$
M$_\odot$ at $z=0$, with one galaxy of each mass experiencing star formation
at the present epoch and one that is relatively quiescent.  These will be compared with
observations of local dwarf galaxies
\cite{2000A&A...355..128C,2011A&A...529A..94C,2012MNRAS.423L.127R,Mao12,2013MNRAS.435..149N,2014A&A...567A.134J}.

\item Two galaxies that are massive ellipticals (having essentially no
star formation) at $z=0$, which will be compared to local ellipticals
\cite{1993A&ARv...4..449W,1996MNRAS.279..229M}.

\end{enumerate}

Once these galaxies are selected from the pilot calculations, we will re-generate the initial
conditions for these galaxies at much higher mass and spatial
resolution using the MUSIC cosmological IC
code~\cite{2011MNRAS.415.2101H}, and then re-run the calculations to
the redshift of interest (typically the present day) at very high
spatial resolution ($\Delta \mathrm{x}_{min} \simeq 100$~pc) using
prescriptions for metal-dependent radiative cooling, star formation
and feedback, and AGN feedback.  This is the state of the art for
physics-rich cosmological galaxy formation simulations, and results in
galaxies with reasonable $z=0$ properties
\cite[e.g.,][]{2012MNRAS.423.1726S,2014MNRAS.444.1518V,2014MNRAS.445..581H}.
In addition, we will include the equations of ideal
magnetohydrodyamics using a cosmologically-motivated seed field
initialized to $B
\simeq 10^{-15}$~G at $z \simeq 100$ when the simulations begin.

One of the virtues of this type of cosmological simulation is that
they are relatively inexpensive (see Section~\ref{sec:comptime_sims}),
which allows us to experiment with variations in physical models to
understand the effect that model choice may have on our results.
Specifically, we will experiment with
the three magnetic field injection scenarios described the introduction to this
section.  

In addition to probing the questions posed in Sections
\ref{sec:objectives} and comparing to the observations described
\ref{sec:observational_comps}, we will
extract the galaxies at an appropriate point in their evolution and re-simulate,
as describe in Section \ref{sec:molcloud_sims}.  This will allow us to have the
most realistic initial and boundary conditions possible for the isolated
simulations.
%measure the strength and morphology of the evolved Milky-Way galaxies to
%ompare to those of the isolated galaxies, which will begin with idealized
%onditions () but, having substantially higher resolution, evolve in a more
%ealistic fashion.  

%As time and resources permit, we will 
%For some
%subset of these galaxies, in addition to our standard calculation we
%will try varying the strength and configuration of the initial seed field (which is unlikely to make a
%difference in larger halos~\cite[e.g.,][]{Xu10,Xu11}, though it is
%unclear if this is true in smaller halos), will experiment with
%magnetic field generation using the Biermann Battery
%\cite[e.g.,][]{Xu08,2014arXiv1408.4161G}, and explore multiple
%models for magnetic field injection from stellar populations and from
%AGN (in particular, varying the total energy and configuration of the
%magnetic fields injected).

%The cosmological simulations alone will allow us to answer several
%questions: How do magnetic fields develop in galaxies as a function of
%cosmic time, mass, formation history, and morphology?  Does the
%initial seed field's strength or configuration ultimately make a
%difference in the properties of observable magnetic fields in
%galaxies?  How are star formation rate and magnetic field properties related over
%cosmic time?
%And, finally, how (and to what distance) are magnetic
%fields communicated into the intergalactic medium, and can the
%inferred intergalactic magnetic fields
%\cite[e.g.,][]{1999ApJ...511...56K,2010Sci...328...73N} come from
%galaxies alone?  With this latter question, and assuming that most
%magnetic fields in the intergalactic medium come from galactic winds,
%we can potentially draw connections between metal absorbers in the
%intergalactic and circumgalactic medium with magnetic fields --
%connections that can be probed by the Jansky VLA, ALMA, and future
%long-wavelength telescopes such as the Square Kilometer Array.  Observational
%diagnostics will be discussed further in Section \ref{sec:observational_comps}


\vspace{-3mm}
\subsubsection{Galactic Simulations}
\label{sec:molcloud_sims}
\vspace{-2mm}

While the cosmological simulations will be able to directly address a
range of observationally-motivated questions about magnetic fields in
galaxies, their relatively limited ($\Delta \mathrm{x} \simeq 100$~pc) spatial  resolution means that they only
 marginally resolve larger molecular clouds, and are far too
coarsely resolved to directly simulate star formation.  To this end,
we will use our cosmological galaxy simulation data as initial
conditions for idealized calculations of isolated disk, elliptical, and
dwarf galaxies.  
For these calculations, we will extract the target galaxies from Section
\ref{sec:cosmo_sims} and embed them in isolated, non-cosmological boxes.  This will allow us to
increase the resolution to $\Delta \mathrm{x} \simeq 1$ pc and follow molecular cloud
formation, as well as more precisely model star formation and feedback
processes, albeit over substantially less than a Hubble time.
Fortunately, it has been observed that Milky-way sized galaxies have 
magnetic field strengths by $z\sim 0.5$ that are comparable to those
at the present day.  This indicates that the growth time
for such fields is quite short, so we will only need to simulate each galaxy for
a few $Gyr$.  

Recently, several groups have begun to explore magnetic fields in full-galaxy
simulations,  from cosmological initial conditions
\citep{Pakmor17}, in isolated, idealized
disks \citep{Rieder16, Rieder17, Butsky17}, and also focusing on large scale structure in
the IGM \citep{Vazza17}.  The proposed work will compliment these studies by
bridging the lengths scales and exploring the magnetic feedback.

With these simulations, we will
explore the relationship between large-scale galactic magnetization
and the properties of individual molecular clouds over a wider cloud
mass scale.
In addition, these
isolated calculations will allow us to explore in greater detail the
effects that varied energy and magnetic field injection mechanisms from stellar
populations (from stellar winds, AGB, and Type Ia and Type II
supernovae) 
have on magnetic field generation, and also on the possible
effect that resolution might have on dynamo amplification of magnetic
fields in both spiral and elliptical galaxies \cite[an effect that was shown to
be important in high-redshift halos;
][]{2010ApJ...721L.134S,2013AN....334..531S}.

In these simulations we will focus resolution on the cold molecular gas
and follow the injection of kinetic energy from supernovae, and the subsequent
dynamo. This will be done for one of each of the four categories of galaxies:
one spiral, one high-redshift galaxy, one dwarf, and one massive elliptical.  The
massive elliptical, having no star formation or molecular content, may be
omitted based on the magnetic activity seen in the cosmological phase. 
Each of these will be treated with three
magnetic feedback routines: the
plain \emph{dynamo-only} evolution, evolving with the field given by the
cosmological simulation; and the two toroidal feedback methods.  
Our work will attempt to incorporate the latest knowledge in simulating the
supernova driven ISM \citep[e.g.][]{Hill12, Shetty12b, Kim15b, Walch15, Padoan16} to the extent it is numerically feasible.  
The thermal
feedback from supernovae
and thermodynamics will target the formation of molecular clouds.  Supernovae
will be tied to star particles in order to be self-consistent, and inject
$10^{51}\rm{erg}$ of energy and a fraction of that in toroidal magnetic energy,
following \citep{Butsky17}, with the fraction depending on the simulation suite.
Thermal energy will be deposited in a sphere containing 60~\msun, following
\citep{Joung06, Hill12}, in order to keep the gas from radiatively cooling on
an unphysically short time scale.
Chemistry and thermodynamics will initially attempt to 
follow the description in the thorough study by \citep{Walch15}, which
simulated $\rm{H}^+, \rm{H}, \rm{H_2}, \rm{C^+}, \rm{CO}$.  This may provide
computationally prohibitive, in which case we will resort to heating and colling
based on tabular interpolation computed with Cloudy \citep{2014ApJS..211...19B, Ferland17}
and a density-based CO map.  
The chemistry is presently contained in the chemistry solver in \enzo\
\cite[see, e.g.,][]{Tasker11,2014ApJ...783...75M}.

The toroidal feedback has been implemented in \enzo\ using the hyperbolic
divergence scheme \citep{Dedner02}.  This will be extended to include the
Constrained Transport scheme \cite{Gardiner05, Collins10}, which conserves the divergence of the magnetic
field to higher precision.  This is theoretically straightforward, but
experience of the team has shown that ensuring field update stays consistent
across multiple refinement levels and in parallel has a number of technical
subtleties that make it unsuitable for an inexperienced graduate student.  In
order to ensure the code development aspect of this project is brief, we are
requesting a postdoc to carry out this work.  

%These simulations will allow us to refine the star formation and feedback
%algorithms by more finely resolving the sampled IMF and more finely resolving
%the turbulence and dynamo action. They will also allow us to explore the
%creation and destruction of molecular clouds in a magnetized environment, which
%has had little study to date \citep{Dobbs08, Kim15}, and the connection between the
%magnetic field of individual clouds to the galaxy itself.  The
%simulations described here will likely require a more detailed
%treatment of H$_2$ formation -- including dust grain formation,
%photoelectric heating, the effect of non-trivial optical depths on
%cooling rates, and others -- which are already contained in the \enzo\
%chemistry solver \cite[see, e.g.,][]{Tasker11,2014ApJ...783...75M}.
%In roughly half of the simulations we will inject magnetic fields from
%supernovae explosions, by adding it in a divergence-free manner as was done in
%\citet{Li06b, Nakamura06, Xu08c}, modeling the injection of magnetic energy into
%the ISM by supernovae.  This  will be compared to
%simulations that have no injection and only allow the small- and large-scale
%dynamos to operate, as in \citep{Beck12}.  



\davedeletenew{
\vspace{-3mm}
\subsubsection{Connecting Star Formation Theory and Galaxy Formation}
\label{sec:connections}
\vspace{-2mm}
}

%\vspace{2mm}
%\noindent\textbf{Connecting Star Formation Theory and Cosmology} 
%\vspace{2mm}

\davedeletenew{
\noindent Simulations of galaxy formation and star formation in a cosmological context
rely, by necessity, on ad-hoc prescriptions for star formation.  For instance,
the oft-used method of \citet{Cen93} requires that material is above a certain
threshold and has negative velocity divergence (i.e., converging gas
flow), then turns a fraction of gas in a resolution element
into stars.  This type of algorithm has a number of free parameters that are
typically normalized by their ability to reproduce observed star formation
properties of star forming galaxies.   Meanwhile, a substantial amount
of work has gone into understanding the rate and mass fraction of star formation
at the molecular cloud (MC) level.  In the proposed work, we will bridge the gap
between the local and cosmologial models by developing a new star formation algorithm that
incorporates developments in molecular cloud dynamics.  Ad-hoc star formation algorithms have done reasonably well  in
reproducing observable quantities such as the redshift evolution of stellar mass in galaxies.  However, the tight, linear
correlation between total gas mass and star formation rate, even at high
redshift, implies that the details of the giant molecular clouds are essential in determining star
formation rates \citep{Genzel10,Shapley11, Lada12, Lada14}.  \textbf{We will
develop a new star formation algorithm for use in cosmological simulations that
takes advantage of molecular cloud physics.}  This development will happen in
conjunction with both molecular cloud and cosmological simulations.
%%%paragraph
Predictions of the mass distribution and rate of star formation can be made
based on the statistics of the turbulence at the molecular cloud level.  We will
follow the pioneering work of \citep{Padoan02, Krumholz05} and the later
extensions by \citep{Hennebelle08c, Padoan11, Hennebelle11, Federrath12} to
formulate a new star formation procedure that incorporates the magnetic field of
the galaxy and molecular cloud properties.
%%%paragraph
One of the most fortuitous events in the study of supersonic turbulence is the
fact that the density probability distribution function (PDF) tends to form a
lognormal.  This has been exploited by several star formation models, and we
will do the same in this project.  The width of the PDF is related in turn to the Mach
number $\mach=v_{\rm{rms}}/c_{\rm{s}}$: higher \mach\ yields higher compression, thus
wider PDFs.  Using the standard turbulent and magnetic scaling laws, one can
formulate an analytic expression based solely on the mean Mach number, magnetic
field strength, and cloud mass.  Several variations of this have been
done, and we
will explore each in turn.  A similar model  was recently developed  with the
code Nyx \citep{Braun15}, wherein it was demonstrated that this technique is
successful in reproducing observed star formation laws.  We will extend this
work by additionally incorporating magnetic fields, in order to directly
incorporate MHD effects into our cosmological simulations
}


\davedelete{
Another fortuitous event is that the width of the
density can be related to the Mach number.
\begin{align}
p(\rho) d\ln \rho &= \frac{1}{\sqrt{2 \pi \sigma^2}} e^{-(\ln\rho -
\overline{\ln\rho})^2/(2 \sigma^2)} d \ln\rho \\
\sigma^2 &= \ln(1 + b^2 \mach^2),
\end{align}
where $\rho$ is the density of gas in units of the mean, $\overline{\ln\rho}$ is
the mean, $\sigma$ is the width of the PDF, and $b$ is a number of order unity.
Experimentally, $b$ ranges from 1/3 for incompressible flow to 1 for fully
compressible flow.  This PDF can in turn be used to predict a star formation
rate by noting that above some threshold, $\rhoc$, a parcel of gas will have a
Jeans mass smaller than the mass of the parcel, and will thus collapse to form
stars.  The effective Jeans mass should incorporate not only the gas pressure,
but also the magnetic pressure and turbulent pressure as well.  Several methods
to incorporate these additional forces will be include in our model,
and details
can be found in \citep{Krumholz05, Padoan11, Hennebelle11, Federrath12}.  
The final rate 
\begin{align}
\rm{SFR} &= \epsilon \int_{\rhoc}^\infty \rho V(\rho) d\rho \\
& = \epsilon \left (1+\rm{erf}\left[\frac{\sigma^2 - 2 \ln \rhoc}{3} \right]
\right),
\end{align}
where SFR is the star formation rate that will be used to determine the mass of
stars in zones.  
The work of \citet{Hennebelle11}  predicts the
initial mass function (IMF) 
from a fragmenting turbulent cloud based on an extension of the
Press-Schechter model to include a lognormal density fluctuation, rather than a
Gaussian, and is further extended to include a Jeans mass that depends on
turbulence and the magnetic field.  This model predicts the IMF from only the properties of the local gas.  
The
difference between this model and other ad-hoc star formation models is that
these parameters are drawn from the physical conditions in the cloud itself,
rather than an adjustable parameter that is set in order to reproduce the
Kennicut-Schmidt law.  
In 
our cosmological simulations, we will be restricted to a
fine-resolution grid of $\simeq 100$ pc,
which is larger than a typical star forming cloud.  Thus in order to model the
typical $\mach$ number and mean magnetic field strength  in our gas, which are
both necessary for the star formation recipe,  we must employ a subgrid scale
model (SGS).  Here we will use a combination of the AMR SGS model
\emph{Fearless}  \citep{Maier09}, which has been implemented in \enzo, and extend
it to include the effects of MHD based on the formulation outlined in
\citep{Chernyshov14}.  In this phase we will only worry about the subgrid
production terms, treating them as a passive quantity as far as the evolution of
the galaxy goes, and feeding them into our equation.
We 
will perform a suite of simulations examining the connection between star
formation on the molecular cloud level and the impact of magnetic fields.  A
significant distinction between the models for cutoff value is the treatment of
magnetic fields.  The work of \citet{Krumholz05} ignores them, while
\citet{Padoan11} treats the additional support of magnetic fields on equal
footing with the turbulence.  Simulations comparing these models will be
performed.
}

\vspace{-3mm}
\subsubsection{Computing time}
\label{sec:comptime_sims}
\vspace{-2mm}

Zoom-in cosmological simulations of a single Milky Way-type galaxy
using \enzo, with $\simeq 400$~pc resolution, require approximately
100,000 CPU-hours on TACC's Stampede resource
\cite{2012ApJ...749..140H,2012ApJ...759..137J,2013MNRAS.430.1548H}, or
slightly less on NCSA's Blue Waters.  The addition of MHD roughly
doubles the computational cost, and increasing the particle mass
resolution by a factor of 8 (to $5 \times 10^5$~M$_\odot$) will help
to resolve early structure formation, but will increase the
computational time by another factor of approximately four (rather
than 8, due to judicious choices made during the creation of initial
conditions that will reduce the overall number of particles).
Increasing spatial resolution will result in approximately a factor of
two increase in cost.  Together, this suggests that a high resolution,
physics-rich Milky Way-type simulation will cost roughly 100,000
node-hours on Stampede2 or Blue Waters -- expensive, but not impossibly
so.  Dwarf galaxy simulations and galaxies that stop at $z \simeq 2$
will require substantially less time.  Isolated disk simulations will
be comparably inexpensive -- at most 2,500 node-hours apiece on either
Stampede2 or Blue Waters.  Taken together, we estimate that we will
need 0.5-1 million node-hours per year on a machine like Stampede2 or
Blue Waters for the cosmological and isolated disk calculations
required for this proposal.

Dr. O'Shea and Dr. Collins have had significant success in procuring
computing time on XSEDE resources as PIs and Co-PIs of numerous large
allocation -- they have over 25 million combined core-hours of XSEDE
computing time over the past five years, most recently on XRAC
allocations TG-AST090040 and AST140008.  In addition, Dr. O'Shea was
the PI of a Blue Waters PRAC allocation that ended in March 2015
consisting of 124 million CPU-hours, and Drs. O'Shea and Collins are
co-PIs of a  Blue Waters allocation (continuing through March 2018) 
of roughly 100 million CPU-hours.    Both PIs will pursue additional
computer time on XSEDE resources and and Blue Waters (or its
progenitor system) that will be devoted primarily to the
project described in this proposal, and the level of resources
obtained is more than adequate for the proposed science.

% \subsubsection{Enzo-P -- maybe}

% \red{do this only if we have time and feel like making this a CDS\& E proposal!}


% Section 3.2 observational comparisons
\vspace{-3mm}
\subsection{Observational Comparisons}
\label{sec:observational_comps}
\vspace{-2mm}

We will perform synthetic observations of simulations using each of the
three magnetic feedback methods described in the previous section, and
will compare them
to a number of recent important observations.  Each comparison will use the
observed data from the literature, synthetic observations employing tools previously
developed by our team and previous collaborators
\citep[e.g.,][]{2013ApJ...765...21S,Barrow17,Barrow17_FL2}, convolved with
resolution and noise appropriate for the target measurement.  We will also
employ ``perfect telescope'' observations directly from the data.  Where
possible the data reduction tools of the particular telescope will be used \citep[e.g.,
CASA; ][]{McMullin07}.
This will allow us to do two things: first, we will verify the physical picture of
our simulations, allowing us to isolate problems or
success with the calculation; second, we will test our basic hypothesis, that
magnetic feedback from supernova explosions is an essential piece of the
magnetization of the universe.  Finally, we will make
predictions for measurements that will be made by both current and
future observatories.  These will primarily target next-generation radio
telescopes and their pathfinders, as well as JWST and WFIRST.

Our catalogue of observations will include the following: synchrotron emission
polarization; Faraday rotation measure and rotation measure synthesis; CO
emission, assuming a conversion from CO-to-H$_2$ appropriate for the system
\citet[e.g.][]{Genzel12,Clark15}; star
formation tracers such as H$\alpha$ and FIR luminosity; and Zeeman Splitting in
dense cores.

\vspace{-3mm}
\subsubsection{Extragalactic Observations}
\vspace{-2mm}

We will compare our simulations to Faraday rotation measures from
observations of high-redshift galaxies
\cite{2008Natur.454..302B,2008ApJ...676...70K,1998A&A...329..809A},
the field magnitude and arrangement in local dwarf galaxies,
\cite{2000A&A...355..128C,2011A&A...529A..94C,2012MNRAS.423L.127R,Mao12,2013MNRAS.435..149N,2014A&A...567A.134J},
field strength, structure and coherence in nearby elliptical
\cite{1993A&ARv...4..449W,1996MNRAS.279..229M} and spiral galaxies
\cite{2014arXiv1411.1386V}.  We will also use measurements of
large-scale intergalactic magnetic fields as a constraint
\cite{2010Sci...328...73N}.   
%RM measurements will include also include the
%use of depolarization of background sources to explore the field strength and
%turbulence of galxies
%\citep{Schulman92}.  
Recently techniques have been developed to combine polarized synchrotron
emission with Faraday Rotation depths to probe the full magnetic configuration
\citep{Heald09, Mao15}.
We will primarily aim to reproduce the morphology of observed integrated
polarization angles of galaxies \citep{Stil09}, the alignment of molecular
clouds and spiral arms \citep{2011Natur.479..499L}, and the relation between
field strength and galactic properties such as mass and velocity dispersion
\citep{2014arXiv1411.1386V,Tabatabaei16}.
We will then make predictions for (and thus
motivate) future observations of galactic magnetic field properties that will be
available to the current and future radio telescopes such as the Jansky VLA, ALMA, and LOFAR, and in the
further future the Square Kilometer Array pathfinder telescopes such
as ASKAP, APERTIF, MeerKAT, and
the SKA itself.

An extremely useful probe of the correlation of magnetic field properties and
star formation is the
correlation between the Far Infrared Radiation (FIR) flux and the
Radio Continuum (RC).  A surprisingly tight correlation between these
two fluxes has been studied for several decades \citep{Helou85}.  Both
radiation sources are indirectly related to star formation, and we
will aim to probe this correlation as a function of redshift.  The FIR
comes from re-heating of dust near sites of massive star formation,
while the RC comes form synchrotron radiation of cosmic rays
\citep{Helou93,Niklas97}.  \cite{Murphy06} model the radio profiles
as smoothed version of the FIR profiles, indicating that for the most
part the RC can be seen as a diffusive flux of cosmic rays.  However,
accurate measurement of this relation has not been reproduced in
cosmological simulations.  Modeling the FIR is done in post-production
after the simulations have finished, by measuring the mass
distribution in stars at a given time and using radiative transfer to
measure the optical depth to FIR.  Modeling the RC requires following
the cosmic ray spectrum and the competition between radiation terms.
These include inverse Compton scattering, ionization of neutral gas,
and bremsstrahlung in addition to the synchrotron that is observable in the RC.  \cite{Schleicher13b} propose that at high
redshift, inverse Compton may begin to dominate over synchrotron and the FIR-RC
correlation will break down around $z= 2-4$, depending on the strength
of the field and the star formation rate.  Several of our simulations
will include the cosmic ray treatment incorporated into Enzo and
described in \cite{Salem14,Salem14b}, suitably modified to incorporate
cosmic ray losses by the aforementioned processes.  This will allow us
to predict future observations by ALMA and LOFAR and to explore the
utility of the RC as a probe of highly extincted star formation.

Another important correlation is the ``Star Formation Main Sequence''
(SFMS) of galaxies \cite{Daddi07, Speagle14}. This relation
between star formation rate and stellar mass is a power law that is
essentially constant over time, with a normalization that falls with
time.  This can be interpreted as a relatively quiescent, steady-state
star formation process with cosmic time.  Recent simulation efforts
have reported good agreement with the SFMS at low redshift, but
reduced star formation rate at intermediate redshifts
\citep{Sparre15}.  We will measure this for not only the highly
resolved galaxies, but also the entire population of galaxies formed
from the cosmological simulation.  This will be compared to both the
low redshift behavior \cite{Brinchmann04} and the high redshift
properties \cite{Daddi07}.  This measurement itself is not a probe of magnetic
fields, but as star formation depends on field strength we will
correlate the slope and normalization of this linear relation with
magnetic field strength in the constituent galaxies.

\vspace{-3mm}
\subsubsection{Galactic Observations}
\vspace{-2mm}


One of the primary goals in this work is understanding the
relationship between the large scale magnetic field
of the galaxy and the magnetic fields in molecular clouds.  As discussed in Section \ref{sec:conngalmcs}, there is clear evidence of
coherent magnetic fields in galactic structure, but the importance in
molecular cloud structure and dynamics is ambiguous.  

The primary question is:
Over what scales are the magnetic field aligned, and how does this alignment
imprint on the molecular gas and impact its dynamics?  To properly answer this
question, we will focus on three measurements, ensuring consistency with earlier
observational work and exploring field properties that have not been measured.
These properties are: the properties of individual molecular clouds;
the correlation between field
directions at two points; and correlations between field direction and material
gradients.  

\noindent
{\bf Molecular Cloud Properties.}
The mean magnetic field strength in molecular clouds is challenging to measure,
but potentially of critical importance in star formation theory.  We will reconcile
molecular cloud properties with observational molecular clouds by comparing
synthetic $^{13}\rm{CO}$ maps with linewidth-size and mass-size relations as
measured by the Galactic Ring survey \citep{Jackson06,Roman-Duval10}.  We will
then predict the mean field strength vs. linewidth and size, and discuss the
expected mass-to-flux and kinetic-to-magnetic energy distributions.

\noindent
{\bf Field-Field Correlation.}
We will use 
synthetic polarization and Faraday rotation measure to measure the correlation
length in the galactic scale simulations.  We will also measure the distribution
of alignments of magnetic fields between and within individual molecular clouds,
and between each molecular cloud and the three-dimensional, kpc-scale
mean field.
These will be compared to the well-correlated alignment seen within molecular
clouds \citep{Li09} and  along galactic spiral arms \citep{Fletcher11}.  This will
allow us to ensure the validity of the fields in our galaxies, and predict the
probability of alignment between any pair of molecular clouds.

\noindent
{\bf Field-Mass Correlation.}
We will also explore the relation between field strength and material.  This has
garnered recent attention as magnetic field tends to align with low density HI
as seen in the GALFA-HI survey  \citep{Clark14}, and tends to lie perpendicular
to higher density structure \citep{PlanckXXXII14}.  We will perform synthetic dust
emission and synthetic HI maps to examine these properties.  Dust polarization
results will be compared to the histogram of relative orientation between
polarization angle and column density gradient. For consistent comparison with
the HI measurements, we will perform the Rolling Hough Transform \citep{Clark14} on the
synthetic HI map to determine the linear structure, and correlate that with the
polarization map.  These measurements will assess the validity of the data.  We
will also explore the alignment of clouds themselves on the large scale
patterns, and measure the probability of finding a cloud at a certain distance
along the large scale field.  This will measure the interaction of magnetic
field and structure on the 100-1000 pc scale.

\noindent
{\bf CMB Foregrounds}
We will also produce all-sky maps for the study of CMB foregrounds.  Maps of the
sky are necessary for training image processing and noise reduction algorithms,
an essential part of the success of future missions.  We will provide a suite of
all-sky maps for such purposes.  We will also measure the variation of
$C_\ell^{EE}$  and $C_\ell^{BB}$ with time and magnetic feedback prescription.
This will allow us to understand the constituent physics that sets $\alpha_{EE}$
and $\alpha_{BB}$.



%\red{CMB foreground stuff goes here}





% Section 3.3: the collaboration
%
\vspace{-3mm}
\subsection{Collaboration Structure}
\label{sec:collaboration}
\vspace{-2mm} 

