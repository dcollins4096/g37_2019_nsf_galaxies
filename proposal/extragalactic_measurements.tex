We will compare our simulations to Faraday rotation measures from
observations of high-redshift galaxies
\cite{2008Natur.454..302B,2008ApJ...676...70K,1998A&A...329..809A},
%the field magnitude and arrangement in local dwarf galaxies,
%\cite{2000A&A...355..128C,2011A&A...529A..94C,2012MNRAS.423L.127R,Mao12,2013MNRAS.435..149N,2014A&A...567A.134J},
the field strength, and the  structure and coherence in nearby 
%elliptical \cite{1993A&ARv...4..449W,1996MNRAS.279..229M} and 
spiral galaxies
\cite{2014arXiv1411.1386V}.  We will also use measurements of
large-scale intergalactic magnetic fields as a constraint
\cite{2010Sci...328...73N}.   
%RM measurements will include also include the
%use of depolarization of background sources to explore the field strength and
%turbulence of galxies
%\citep{Schulman92}.  
Recently techniques have been developed to combine polarized synchrotron
emission with Faraday Rotation depths to probe the full magnetic configuration
\citep{Heald09, Mao15}.
We will primarily aim to reproduce the morphology of observed integrated
polarization angles of galaxies \citep{Stil09}, the alignment of molecular
clouds and spiral arms \citep{2011Natur.479..499L}, and the relation between
field strength and galactic properties such as mass and velocity dispersion
\citep{2014arXiv1411.1386V,Tabatabaei16}.
We will then make predictions for (and thus
motivate) future observations of galactic magnetic field properties that will be
available to the current and future radio telescopes such as the Jansky VLA, ALMA, and LOFAR, and in the
further future the Square Kilometer Array pathfinder telescopes such
as ASKAP, APERTIF, MeerKAT, and
the SKA itself.

An extremely useful probe of the correlation of magnetic field properties and
star formation is the
correlation between the Far Infrared Radiation (FIR) flux and the
Radio Continuum (RC).  A surprisingly tight correlation between these
two fluxes has been studied for several decades \citep{Helou85}.  Both
radiation sources are indirectly related to star formation, and we
will aim to probe this correlation as a function of redshift.  The FIR
comes from re-heating of dust near sites of massive star formation,
while the RC comes form synchrotron radiation
\citep{Helou93,Niklas97}.  \cite{Murphy06} model the radio profiles
as smoothed version of the FIR profiles, indicating that for the most
part the RC can be seen as a diffusive flux of cosmic rays.  However,
accurate measurement of this relation has not been reproduced in
cosmological simulations.  Modeling the FIR is done in post-production
after the simulations have finished, by measuring the mass
distribution in stars at a given time and using radiative transfer to
measure the optical depth to FIR.  Modeling the RC requires following
the cosmic ray spectrum and the competition between radiation terms.
These include inverse Compton scattering, ionization of neutral gas,
and bremsstrahlung in addition to the synchrotron that is observable in the RC.  \cite{Schleicher13b} propose that at high
redshift, inverse Compton may begin to dominate over synchrotron and the FIR-RC
correlation will break down around $z= 2-4$, depending on the strength
of the field and the star formation rate.  Several of our simulations
will include the cosmic ray treatment incorporated into Enzo and
described in \cite{Salem14,Salem14b}, suitably modified to incorporate
cosmic ray losses by the aforementioned processes.  This will allow us
to predict future observations by ALMA and LOFAR and to explore the
utility of the RC as a probe of highly extincted star formation.

Another important correlation is the ``Star Formation Main Sequence''
(SFMS) of galaxies \cite{Daddi07, Speagle14}. This relation
between star formation rate and stellar mass is a power law that is
essentially constant over time, with a normalization that falls with
time.  This can be interpreted as a relatively quiescent, steady-state
star formation process with cosmic time.  Recent simulation efforts
have reported good agreement with the SFMS at low redshift, but
reduced star formation rate at intermediate redshifts
\citep{Sparre15}.  We will measure this for not only the highly
resolved galaxies, but also the entire population of galaxies formed
from the cosmological simulation.  This will be compared to both the
low redshift behavior \cite{Brinchmann04} and the high redshift
properties \cite{Daddi07}.  This measurement itself is not a probe of magnetic
fields, but as star formation depends on field strength we will
correlate the slope and normalization of this linear relation with
magnetic field strength in the constituent galaxies.
