\vspace{-4mm}
\section{Significance of Proposed Work}
\vspace{-3mm}

Magnetic fields are essential components of galactic structure, but
the nature of their growth and morphology is elusive to understand
from a theoretical and numerical standpoint.  The challenge in doing
so lies in the difficulty of devising a consistent treatment of
initial and boundary conditions, of including the necessary physics,
and in interpreting observations.  The proposed project will model the
origin and evolution of galactic magnetic fields, and their impact on
star-forming molecular clouds, from cosmological initial conditions to
the present day.  We will also create synthetic observations for JVLA,
ALMA, and other telescopes, which will be crucial to interpretating
existing and upcoming observations.  In addition, we will use these
tools to explore the possibilities of future observations that will be
made by the next generation of radio telescopes  (in particular, the ngVLA and SKA).  This work will
result in a deep understanding of the connection between galaxies and
molecular clouds, and of the magnetic fields that permeate these
objects.  This project is innovative due to its connection of
cosmological structure (i.e., galaxies and their environments) to
molecular clouds and the star formation that occurs therein, and the
modeling of both of these classes of objects in a unified theoretical
and numerical framework.  The understanding gained by doing this will
be transformative in terms of our ability to more accurately model
star formation through cosmic time, and will directly connect to
current and future observations of magnetic fields in a range of
astrophysical situations.
