%%%%%%%%%%%%%%%%%%%%%%%%%%%%%%%%%%%%%%%%%%%%%%%%%%%%%%%%%%%%%%%%%%%%%%%%%%%%
\vspace{-4mm}
\section{Intellectual merit and broader impacts}
\vspace{-3mm}

\textbf{Intellectual merit:} This project is novel because it will use
high resolution magnetohydrodynamics simulations to self-consistently
follow the evolution of plasma over a huge range of astrophysically
important length, density, and temporal scales, thus bridging the gap
between the cosmological structure formation that results in
galactic-scale magnetic fields and the star-forming molecular clouds
that form out of the magnetized interstellar medium.  This is crucial
because an improved understanding of how magnetized molecular clouds
form within galaxies will lead to more accurate initial conditions for
targeted studies of star formation, and will provide an opportunity to
model that critical process in a more realistic way.  The results of
this project will facilitate our interpretation of observations of
magnetic fields in the intergalactic medium, in both high redshift and
nearby galaxies, and in the Milky Way galaxy itself.

\vspace{1mm}

\noindent\textbf{Broader impacts:} Our proposed work will have significant
impact on scientists in training, who will learn to use cutting-edge numerical
tools at the largest possible scale and will develop critical skills in
scientific software development and data analysis.  All of the tools developed
as part of this work will be incorporated into widely used open-source software
projects and all simulation and analysis data products will be made publicly
available.  This will maximize the return on this investment by enabling the
community to more easily build upon this work.  In addition, the result of these
studies (and other astrophysics knowledge) are disseminated through ``Ask a
Scientist,'' a public outreach event we hold at
Tallahassee's monthly art festival. 
