%%%%%%%%%%%%%%%%%%%%%%%%%%%%%%%%%%%%%%%%%%%%%%%%%%%%%%%%%%%%%%%%%%%%%%%%%%%%
\vspace{-4mm}
\section{Intellectual merit and broader impacts}
\vspace{-3mm}

\textbf{Intellectual merit:} This project is novel because it will use
high resolution magnetohydrodynamics simulations to self-consistently
follow the evolution of plasma over a huge range of astrophysically
important length, density, and temporal scales, thus bridging the gap
between the cosmological structure formation that results in
galactic-scale magnetic fields and the star-forming molecular clouds
that form out of the magnetized interstellar medium.  This is crucial
because an improved understanding of how magnetized molecular clouds
form within galaxies will lead to more accurate initial conditions for
targeted studies of star formation, and will provide an opportunity to
model that critical process in a more realistic way.  The results of
this project will facilitate our interpretation of observations of
magnetic fields in the intergalactic medium, in both high redshift and
nearby galaxies, and in the Milky Way galaxy itself.

\vspace{1mm}

\noindent\textbf{Broader impacts:} Our proposed work will have
significant impact on scientists in training, who will learn to use
cutting-edge numerical tools at the largest possible scale and will develop
critical skills in scientific software development and data analysis.
We will involve undergraduate and graduate students at MSU and a
postdoctoral researcher at FSU in our research efforts (with undergraduates recruited through
MSU's Research Experience for Undergraduates program), including
students from under-represented groups.  All of the tools developed as
part of this work will be incorporated into widely used open-source
software projects and all simulation and analysis data products will
be made publicly available.  This will maximize the return on this
investment by enabling the community to more easily build upon this
work.  Scientific results from this program will be visualized by
members of our collaboration, and will be disseminated to the public
via our pre-existing collaborations with planetaria and museums, and
via the Internet.  In addition, these visualizations will be used as
part of outreach talks and ``Ask a Scientist'' events that involve
members of this project.  The simulation data produced as a result of
this project will be used in computational science courses at Michigan
State, where it will be used to train students in scientific
visualization and data analysis techniques.  Similarly, this data will
be used at Florida State in undergraduate-level ``Hydrodynamics for
astrophysics'' and ``Extragalactic astrophysics'' courses, and will
use the results obtained in this project to incorporate active
research into the education of undergraduate physics and astrophysics
majors.  The resulting curricular materials will be made available to
the public via the Internet.

