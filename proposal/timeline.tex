%%%%%%%%%%%%%%%%%%%%%%%%%%%%%%%%%%%%%%%%%%%%%%%%%%%%%%%%%%%%%%%%%%%%%%%%%%%%
\vspace{-4mm}
\section{Collaboration management plan and timeline}
\label{sec:timeline}
\vspace{-3mm}

\textbf{The Team.}  
Dr. Brian O'Shea (PI) is an expert in cosmological simulations of
structure formation, and has modeled galaxy evolution from the
earliest galaxies through present-day galaxy clusters.  He is an
expert on the sub-grid modeling of astrophysical processes in the
\enzo\ code, and is wholly or partially resonsible for several of
\enzo's most critical modules for this project, including the star
formation and feedback and AGN feedback algorithms.  In this work,
Dr. O'Shea and his research group at Michigan State University will be
responsible for the cosmological simulations of galaxy formation, and
will lead the design and analysis of idealized Milky Way and dwarf
galaxy simulations.

Dr. David Collins (Co-PI) is an expert in numerical MHD methods and
simulations of star forming clouds, and has participated in
simulations of MHD turbulence in a variety of contexts, including
first stars, and galaxy clusters.  He was primarily responsible for
the development of the MHD solver that will be used for this work.  In
the proposed work, Dr. Collins and his research group at Florida State
University will be responsible for simulations at the galaxy scale and
below, focusing on the formation and collapse of molecular clouds, and
the strength and topology of the galactic magnetic field.  They will
also be responsible for the development of the new MHD-based star
formation and feedback algorithm, and will lead the design and
analysis of idealized disk galaxy and ISM/molecular cloud simulations.

Dr. Sui Ann Mao (external collaborator; group leader at the Max Planck
Institute for Radio Astronomy in Bonn, Germany; see attached letter of
collaboration) is an expert in the
measurement of magnetic field structure in the Milky Way, the Large
Magellanic Cloud, 
Local Volume galaxies,
and in more distant galaxies. She will offer
guidance regarding synthetic observations of simulations at all
scales, suggest additional measurements beyond those described in this
proposal, and will verify that the behavior of simulated fields in the
simulations described in this proposal are consistent with current
observations.  Our work will also inform her observational campaign by
predicting observational quantities.

\vspace{2mm}
\noindent
\textbf{Plan and Timeline.}
PI O'Shea (MSU) and co-PI Collins (FSU) jointly have responsiblity for
carrying out the research described in this proposal and for reporting
progress and results to the NSF.  They individually are responsible
for overseeing the grant budget and for mentoring the graduate student
at MSU (O'Shea) and the postdoctoral researcher at FSU (Collins).

At the scientific and technical level, Dr. O'Shea and the MSU graduate
student are responsible for designing, running, and analyzing the
cosmological galaxy formation simulations and, in collaboration with
Dr. Mao, comparing the simulations to extragalactic observations of
magnetic fields.  The MSU and FSU contingents will work together to
extract initial conditions for the galaxy-level simulations from
cosmological calculations, and the FSU group will focus on the
study of magnetic field evolution in isolated galaxy simulations under
a variety of circumstances and on the formation of molecular
cloud-based star formation and feedback algorithms.  Both the FSU and
MSU research groups, as well as Dr. Mao, will work together to further
develop the synthetic observing capabilities in \yt\ as they pertain
to magnetic field observations, and on comparing simulations with
observations at all physical scales.

The timetable and milestones for this project are listed below.  We
assume for planning purposes that the work done as a part of this
project will start in August 2018.

\vspace{-2mm}

\begin{itemize}
\item{\textbf{Year 1} (2018-19): The postdoc will implement the feedback in the
CT solver.  In parallel, the run first round of cosmological MHD
zoom-in simulations of all galaxies will be run with the existing feedback.  Extract initial conditions for
first isolated galaxy calculations.  Begin data analysis of
cosmological simulations.  Run first isolated galaxy simulations.  
{\em Milestones}: Complete first round of cosmological simulations.  }

\item{\textbf{Year 2} (2019-20): Finish data analysis of first round
of cosmological simulations.  Run first round of isolated galaxy
calculations, informing improved subgrid star formation and feedback
models.  Run and analyze zoom-in cosmological simulations with physics
variations.  Analyze molecular cloud simulations and create improved
subgrid stellar feedback model.  {\em Milestones}: Completed data
analysis for first set of cosmological simulations and first round of
isolated galaxy simulations.  Creat subgrid stellar feedback model
with MHD.  }

\item{\textbf{Year 3} (2020-21): Run and analyze final round of
zoom-in cosmological simulations and isolated galaxy simulations,
using new subgrid MHD stellar feedback model.   {\em
Milestones}: Completed final round of cosmological simulations and
isolated galaxy simulations (including data analysis). }

\item{\textbf{End of grant}: Public release of all new \enzo\ modules,
as well as all \yt\ synthetic observation tools and example scripts.
Public release of all simulation data and data products.}

\end{itemize}

\vspace{-3mm}
