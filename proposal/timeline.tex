%%%%%%%%%%%%%%%%%%%%%%%%%%%%%%%%%%%%%%%%%%%%%%%%%%%%%%%%%%%%%%%%%%%%%%%%%%%%
\vspace{-4mm}
\section{Management plan and timeline}
\label{sec:timeline}
\vspace{-3mm}

\textbf{The Team.}  

Dr. David Collins (Co-PI) is an expert in numerical MHD methods and
simulations of star forming clouds, and has participated in
simulations of MHD turbulence in a variety of contexts, including
first stars and galaxy clusters.  He was primarily responsible for
the development of the MHD solver that will be used for this work.  In
the proposed work, Dr. Collins will be responsible for directing the activity,
analysis of simulation products, and code development as necessary.


%University will be responsible for simulations at the galaxy scale and
%below, focusing on the formation and collapse of molecular clouds, and
%the strength and topology of the galactic magnetic field.  They will
%also be responsible for the development of the new MHD-based star
%formation and feedback algorithm, and will lead the design and
%analysis of idealized disk galaxy and ISM/molecular cloud simulations.

Dr. Brian O'Shea (external collaborator, Michigan State University) is an expert in cosmological simulations of
structure formation, and has modeled galaxy evolution from the
earliest galaxies through present-day galaxy clusters.  He is an
expert on the sub-grid modeling of astrophysical processes in the
\enzo\ code, and is wholly or partially responsible for several of
\enzo's most critical modules for this project, including the star
formation and feedback and AGN feedback algorithms.  In this work,
Dr. O'Shea and his research group at Michigan State University will be
responsible for the remaining \enzoe\ developments and interpretation of scientific results.

The proposal will also include one graduate student at FSU.  This student will
be responsible for performance of both cosmological and isolated galaxy
simulations and analysis of results.

\vspace{2mm}
\noindent
\textbf{Plan and Timeline.}

PI Collins will be responsible for
carrying out the research described in this proposal and for reporting
progress and results to the NSF, and will be responsible
for overseeing the grant budget and for mentoring the graduate student.

Dr. O'Shea and the team at MSU are responsible for the final development needs
of \enzoe, and will aid in the interpretation and analysis of results.

The proposal will provide support for one unnamed graduate student.  As the
student has not been selected, we will structure the timeline assuming an
inexperienced student.  

The timetable and milestones for this project are listed below.  We
assume for planning purposes that the work done as a part of this
project will start in August 2020.
This project runs the risk of being overly ambitious.  
New code development 
brings uncertainties, and there are more interesting questions to be asked of
the resulting simulations than we can possibly ask with the personnel involved.  Here we describe a minimal and conservative work
plan to ensure success of the primary measurements.  To hedge against code
errors and development setbacks, the initial phases will be run with both Enzo
and \enzoe.  This will result in some additional cost as the number of
simulations and their setup will increase, but the redundancy will ensure a
successful results while verifying results from the new code.  Once development
of \enzoe\ is finished, it will be used exclusively.  In the unlikely event
of severe delays, Enzo will be used for the production simulations. 

The basic sequence of simulations for our research plan is as follows:
\begin{enumerate}
    %\item Get the magnetic feedback working with CT (maybe ignore that)
    \item Perform a galaxy formation simulation at cosmological scales
    \item Extract galactic properties from a target galaxy
    \item Simulate isolated galaxy
    \item Produce synthetic observations
\end{enumerate}
Each of these actions uses machinery that has been well tested in \enzo\ and
\yt.  
Using \enzo\ and simple physics and low resolution (e.g. no MHD or supernovae
and small enough to fit on a desktop), these actions can be done by an
inexperienced graduate student in 
6 months to one year.   Using scientifically interesting resolution increases
the simulation and analysis time substantially, as well as the time to solve problems that may
arise.  Similarly, including scientifically interesting physics destabilizes the code
and requires more time of both the computer and the student.  

The time to
perform this sequence of simulations with \enzoe\ is not as well determined, as
many of the tools that make these simulations easy with \enzo\ do not exist for
\enzoe.  However in principle they should have comparable time scales.
The development of \enzoe\ is proceeding concurrently at MSU, and
should be completed June 2020.
%We thus anticipate the new graduate student can perform the full suite of
%high-resolution simulations in the first two years.  In the third year, we will
%produce the final synthetic CMB maps and other synthetic observations.

We will spend the \emph{first} year running a low-resolution version of this sequence
for code development and training purposes.  The \emph{second} year we will begin the
simulations at the target resolution with \enzoe, as well as analyzing the
magnetic history and synthetic maps.  The \emph{third} year we will further explore
variations in mass, by exploring other galaxies in the Tempest simulations.

Conservatively, we will structure our work in the following way:
%\vspace{-2mm}
\begin{itemize}
\item{\textbf{Year 1} (2020-2021): 
    During the first year, the graduate student performs the full sequence of simulations and analysis
    at low resolution, with limited physics, using \enzo.  We will select one of the Tempest
    galaxies to re-simulate with simple MHD at low resolution, extract the
    galactic structure at $z=0$, re-simulate the galaxy at low resolution, and
    produce synthetic dust polarization maps.  We will analyze magnetic field history of
    target galaxy.  We will also begin \enzoe\ on the cosmology simulations as
    the code becomes available.  By the end of the first year we will have the
    complete analysis package for a single low-resolution galaxy.  

{\em Milestones}: Produce low resolution versions of both cosmology and galaxy
simulations with \enzo.  Analyze magnetic history and synthetic polarization maps.  Begin using \enzoe.}

\item{\textbf{Year 2} (2021-2022): 
    Begin a high-resolution sequence of simulations with the same target galaxy
    as Year 1, and all three magnetic field prescriptions.  This will be done
    with a combination of \enzo\ and \enzoe\ depending on the readiness of the
    latter. The high-resolution cosmology simulations will finish and we can
    analyze and publish the cosmological portion of the assembly.

{\em Milestones}: 
Science-ready simulations with three magnetic prescriptions will be performed. Publication of the
magnetic history of the galaxy over cosmic time.  Begin the high-resolution
isolated galaxy runs.
}

\item{\textbf{Year 3} (2022-2023): 
    During the final year we will finish the galactic simulations and work on publication of the synthetic
    observations. We will study the magnetic assembly of the isolated galaxy,
    and compare to that of the cosmological simulations.

{\em Milestones}: Finish and publish galactic simulations.  Release of synthetic observations.  
    }

\item{\textbf{End of grant}: Public release of all new \enzo\ and \enzoe\ modules,
as well as all \yt\ synthetic observation tools and example scripts.
Public release of all simulation data and data products.}

\end{itemize}

\vspace{-3mm}
