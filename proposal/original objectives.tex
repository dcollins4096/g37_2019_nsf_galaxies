\section{Objectives and Summary}

\red{Brian's new version}

The universe is threaded with magnetic fields, which permeate
galaxies, molecular clouds, and stars, and which connect these objects
together in complex ways.  Observations show that galaxies at $z
\simeq 2$ (one-third of the present age of the universe) have magnetic
fields that are both organized on large scales and of substantial
magnitude.  Organized, dynamically important magnetic fields are also
ubiquitous in the interstellar medium of present-day galaxies.  The
molecular clouds that are the sites of star formation in nearby
galaxies form out of this magnetized plasma, and measurements show
that these clouds have substantial magnetic fields as well.  These
magnetic fields are possibly quite important for star formation
\bwo{low-mass and massive, or just low-mass?  or just massive?}, which
remains one of the most important unsolved problem in astrophysics.
While magnetic fields have been measured in all of these astrophysical
regimes, and there is a clear sequence of events -- galaxies form
molecular clouds, which in turn are the sites of star formation -- the
ways in which magnetic fields tie galaxies to molecular clouds, and
thus potentially affect star formation and the initial stellar mass
function, are poorly understood theoretically.  \textbf{There is a
clear need for a detailed, self-consistent model of cosmological
structure formation that can trace the evolution of magnetized gas to
the physical scales relevant for star formation.}

Our long-term goal is to create a predictive model for the evolution
of magnetic fields in the universe, from intergalactic scales down to
that of individual stars, and self-consistently understanding both the
effects of magnetic fields on star formation as well as the ways in
which magnetic fields are injected from stars into their host
galaxies.  Our objective in this proposal is to understand how
galactic-scale magnetic fields affect the formation and evolution of
molecular clouds, and to identify the characteristics of these
magnetized clouds as it pertains to star formation.  Our rationale is
that an improved understanding of how magnetized molecular clouds form
within galaxies will lead to more accurate initial conditions for
targeted studies of star formation, and will provide an opportunity to
model that critical process in a more realistic way.  The team we have
assembled contains experts in magnetohydrodynamic (MHD) simulations of
cosmological structure formation and of star formation, is united in
our use of a sophisticated and high-dynamic-range numerical tool (the
\enzo\ code) that can seamlessly connect the cosmological and stellar
scales, and is strongly motivated by both cosmological and local
observations of magnetic fields.

In this project, we will use high resolution MHD simulations of
cosmological structure formation, along with a sub-grid model for star
formation, feedback, and the injection of magnetic fields from stellar
populations into the interstellar medium, to model the evolution of
magnetic fields over the age of the universe.  These simulations will
provide the initial conditions for simulations of the
interstellar medium in sub-kiloparsec volumes, where we will follow
the formation of magnetized molecular clouds and determine the
properties of the star-forming cores within these clouds.  Finally, we
will model how feedback from stellar populations injects magnetic
fields back into the interstellar medium, potentially affecting the
behavior of the large-scale magnetic fields in galaxies.

The end result of this project will be a deep understanding of the
connection between galaxies and molecular clouds, and of the
properties of the magnetic fields that permeate these objects.  This
project is innovative due to its connection of cosmological structure
(i.e., galaxies) with molecular clouds and the star formation that
occurs therein, and the modeling of both of these classes of objects
in a unified theoretical and numerical framework.  The understanding
gained by doing this will be transformative in terms of our ability to
more accurately model star formation, and will directly connect to
current and future observations of magnetic fields in a range of
astrophysical situations.


\red{old text}

Scientific goals:
Connect star formation to galaxy formation: use cosmological simulations to
determine the properties of molecular clouds (particularly magnetic structure),
and then use those molecular clouds as initial conditions for simulations of
star formation (massive star formation?  solar-mass star formation?  both? we
should get a whole range of GMCs…)

It’s impossible to simulate this whole range of scales (cosmic web to
protostellar cores) in a cyclic way (i.e., include all of them in a simulation
lasting a ~Hubble time), so validate sub-grid models for star formation and
feedback that are used at the cosmological/isolated disk simulation scale.

Do first simulations of cosmological galaxy formation using MHD (first?  maybe
not…)(Dolag beat us.)

\citep{Beck12}

We will do cosmology with magnetic fields and see what the galaxies look like.

We will make some star formation things that connect to modern star formation
models, also that include magnetic fields.

We will look at the morphology of the magnetic fields of galaxies.  They'd
better look right.

We will extract some clouds and do some crazy zoom-ins.  This will get us the
right intiial and boundary conditions for mass, magnetic fields, and kinetic
energy.  Dobbs (last week) showed that this might be important.

We will try to see if there's an obvious scale at which the field decouples from
larger/smaller scales.

Our plan is to bridge the theoretical gap between cosmological structure
formation and the formation of stars in Galactic molecular clouds, including a
self-consistent treatment of magnetic fields from cosmological initial
conditions.”  Understanding magnetic field structure at the Galactic scale is
critical for realistic (need better word) simulations of molecular clouds, and
simulating both allows us to constrain our models by both observations of the
interstellar medium in the Milky Way and distant galaxies, and also by examining
molecular clouds in the Local Group. 


