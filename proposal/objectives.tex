\vspace{-6mm}
\section{Objectives and Summary}
\label{sec:objectives}
\vspace{-3mm}

Observations show that galaxies at $z
\simeq 2$ (one-third of the present age of the universe) have magnetic
fields that are both organized on large scales and of comparable
magnitude to what we see in the Milky Way today.  Organized, dynamically important magnetic fields are also
ubiquitous in the interstellar medium of present-day galaxies.  The
molecular clouds that are the sites of star formation in nearby
galaxies form out of this magnetized plasma, and measurements show
that these clouds have substantial magnetic fields as well.  These
magnetic fields are possibly quite important for star formation, which
remains one of the most important unsolved problems in astrophysics.
While magnetic fields have been observed in all of these astrophysical
regimes, and there is a clear sequence of events -- galaxies form
molecular clouds, which in turn are the sites of star formation -- the
ways in which magnetic fields tie galaxies to molecular clouds, and
thus potentially affect star formation and the initial stellar mass
function, are poorly understood theoretically.  
\textbf{There is a
clear need for a detailed, self-consistent model of cosmological
structure formation that can trace the evolution of magnetized gas to
the physical scales relevant for star formation.}

Our long-term goals are to create a predictive model for the evolution
of magnetic fields in the universe, from intergalactic scales down to
that of individual stars, and to self-consistently understand both the
effects of magnetic fields on star formation as well as the ways in
which magnetic fields are ejected from stars into their host galaxies
and beyond.
\textbf{Our objective in this proposal is to understand how galactic-scale
magnetic fields affect the formation and evolution of molecular
clouds, and in turn how magnetic fields ejected from stars are
amplified and ordered within galaxies and their environments.}  Our rationale is that an improved
understanding of how magnetized molecular clouds form within galaxies
will lead to more accurate initial conditions for targeted studies of
star formation, and will provide an opportunity to model that critical
process in a more realistic way.  
Specifically, we will answer these key questions about magnetic fields
in the universe:
%We will test 
%this hypothesis using high dynamic range MHD simulations
%of
%cosmological structure formation, idealized galaxies, and the
%interstellar medium, along with newly developed subgrid models for
%star formation, feedback and the injection of magnetic
%fields from stellar populations into the interstellar medium.  We will
%compare our theoretical results to galactic and extra-galactic
%observations, taking care to quantify limitations due to observational constraints
%and numerical effects.

\vspace{-2mm}
\begin{enumerate}

\item What are the physical processes that are primarily responsible for the
magnetization of the universe, as a function of galaxy mass and
redshift?  And, how does this lead to the amount of structured
vs. unstructured magnetic field that we see in galaxies today?

\item What is the origin of the intergalactic magnetic fields in the
  cosmic web, and how did those fields get to be where they are today?

\item In Milky Way-like galaxies, how do magnetic fields on different
  physical scales (i.e., CGM, spiral arm, molecular cloud,
  protostellar cores) relate to each other, and what are their
  predicted properties?

\item In Milky Way-like galaxies, what is the relationship between the
  magnetic fields in the hot interstellar medium and in molecular
  clouds, both during cloud formation and after the clouds are
  dispersed by radiation and supernovae?

\end{enumerate}
\vspace{-2mm}

\textbf{We hypothesize that feedback of magnetic fields from
stellar explosions, and subsequent dynamo activity, is responsible for
magnetizing the universe.}  We will address these questions within this
framework by:
\vspace{-2mm}
\begin{enumerate}

\item Performing a suite of galaxy formation simulations from cosmological
initial conditions, using a set of three subgrid models for magnetic field
creation
\item Re-simulate a selection of galaxies at sub-parsec resolution to connect the
magnetic fields to the initial and boundary conditions of their birth.

\item Compare synthetic observations of galaxies at each stage to observed radio
and infrared observations.  With these synthetic observations we will address
our hypothesis, and provide predictions for future observations.


\end{enumerate}
\vspace{-2mm}
% The team we have assembled contains
% experts in magnetohydrodynamic (MHD) simulations of cosmological
% structure formation and of star formation, is united in our use of a
% sophisticated and high-dynamic-range numerical tool that can
% seamlessly connect the cosmological and stellar scales (the \enzo\
% code), and is strongly connected to observations of both extragalactic and local
% magnetic fields through our collaborators.

% to model
% the evolution of magnetic fields over the age of the universe.  These
% calculations will provide the initial conditions for simulations of
% the interstellar medium in idealized, isolated galaxies, where we will
% follow the formation of magnetized molecular clouds and determine the
% impact of magnetic fields on their lifetime, evolution, and bulk star
% formation properties.  Finally, we will model how feedback from
% stellar populations injects magnetic fields back into the interstellar
% medium, potentially affecting the behavior of the large-scale magnetic
% fields in galaxies.  

% Care will be taken to quantify limitations due to observational
% constraints (noise, sensitivity, projection effects) and numerical effects
% (resolution and input assumptions).


The end result of this project will be a deep understanding of the
connection between galaxies and molecular clouds, and of the
properties of the magnetic fields that permeate these objects.  This
project is innovative due to its connection of cosmological structure
(i.e., galaxies and their environments) to molecular clouds and the
star formation that occurs therein, and the modeling of both of these
classes of objects in a unified theoretical and numerical framework.
The understanding gained by doing this will be transformative in terms
of our ability to more accurately model star formation through cosmic
time, and will directly connect to current and future observations of
magnetic fields in a range of astrophysical situations.


