
In addition to existing in galaxies over a wide span of time, the
presence of magnetic fields has been inferred on cosmological scales
in the intergalactic medium.  At megaparsec scales,
\cite{2010Sci...328...73N} show that the lack of measured GeV
gamma-rays in the direction of TeV gamma-ray sources observed by the
Fermi Large Area Telescope yields a \textit{lower bound} of B~$\geq 3
\times 10^{-16}$~G, with the lower bound increasing as $\propto
\lambda_B^{-1/2}$ if the magnetic field correlation length $\lambda_B$
is significantly smaller than a megaparsec.  Different analyses of
data from the same instrument by \cite{2011ApJ...733L..21D,
2010MNRAS.406L..70T}, with varied assumptions, results in lower limits
for the strength of large-scale magnetic fields in the low redshift
intergalactic medium that range from $\geq 10^{-18}$ G to $\geq 5
\times 10^{-15}$~G.  None of these observations provides direct
measurements of the coherence lengths of the fields in question,
though they are inferred to be on the order of megaparsecs or greater.
Also, it should be noted that these are lower limits on the large-scale intergalactic
magnetic field and not a direct measurement.  An upper limit can also be determined by using the
cosmic microwave background (CMB) -- specifically, measurements of CMB
temperature anisotropies and polarization put strong upper limits on a
mean pre-recombination magnetic field strength $\simeq 10^{-9}$~G
\cite{1997PhRvL..78.3610B,2012SSRv..166...37W,2013A&ARv..21...62D}.
We note that the CMB-derived
upper limit is more stringent than the limits
coming from Big Bang Nucleosynthesis, where measurements of the ratios
of primordial species (including H, D, and He), as well as the
baryon-to-photon ration $\eta$, give a strong upper limit for
primordial magnetic field strengths of $10^{-7}$~G.

Intergalactic magnetic fields on smaller scales have also been
observed.  Rotation measure observations of the intracluster medium
-- that is, the intergalactic plasma in clusters of galaxies -- show
that it is threaded by magnetic fields that are on the order of
$0.1-10$~$\mu$G (depending on cluster location)
\cite{Carilli02,2005mpge.conf..231E, 2005A&A...434...67V}, with
patches of much higher magnetization ($\sim 40$~$\mu$G) observed.
Furthermore, ``cool core'' clusters tend to have stronger fields by a
factor of roughly 2 than non-cool core clusters.  The same
observations show that magnetic fields are not regularly ordered on
cluster scales, but instead have coherence lengths that range from a
few kpc to tens of kpc. It is very difficult to discern the volume
filling fraction of magnetic fields in the intracluster medium due to
the paucity of background sources that can be used to observe rotation
measures -- however, essentially all background sources show some
magnetization.
