
\vspace{-3mm}
\subsection{Observational Comparisons}
\label{sec:observational_comps}
\vspace{-2mm}

We will perform synthetic observations of simulations using each of the
three magnetic feedback methods described in the previous section, and
will compare them
to a number of recent important observations.  Each comparison will use the
observed data from the literature, synthetic observations employing tools previously
developed by our team and previous collaborators
\citep[e.g.,][]{2013ApJ...765...21S,Barrow17,Barrow17_FL2}, convolved with
resolution and noise appropriate for the target measurement.  We will also
employ ``perfect telescope'' observations directly from the data.  Where
possible the data reduction tools of the particular telescope will be used \citep[e.g.,
CASA; ][]{McMullin07}.
This will allow us to do two things: first, we will verify the physical picture of
our simulations, allowing us to isolate problems or
success with the calculation; second, we will test our basic hypothesis, that
magnetic feedback from supernova explosions is an essential piece of the
magnetization of the universe.  Finally, we will make
predictions for measurements that will be made by both current and
future observatories.  
%These will primarily target next-generation radio
%telescopes and their pathfinders, as well as JWST and WFIRST.

Our catalogue of observations will include the following: synchrotron emission
polarization; thermal dust emission and its polarization; Faraday rotation
measure and rotation measure synthesis; CO emission, assuming a conversion from
CO-to-H$_2$ appropriate for the system \citet[e.g.][]{Genzel12,Clark15}; and the
far-infrared (FIR) luminosity.%;
%star %formation tracers such as H$\alpha$ and FIR luminosity; and Zeeman
%Splitting in %dense cores.
