\vspace{-3mm}
\subsection{Observational Comparisons}
\label{sec:observational_comps}
\vspace{-2mm}

We will perform synthetic observations of simulations using each of the
three magnetic feedback methods described in the previous section, and
will compare them
to a number of recent important observations.  Each comparison will use the
observed data from the literature, synthetic observations employing tools previously
developed by our team and previous collaborators
\citep[e.g.,][]{2013ApJ...765...21S,Barrow17,Barrow17_FL2}, convolved with
resolution and noise appropriate for the target measurement.  We will also
employ ``perfect telescope'' observations directly from the data.  Where
possible the data reduction tools of the particular telescope will be used \citep[e.g.,
CASA; ][]{McMullin07}.
This will allow us to do two things: first, we will verify the physical picture of
our simulations, allowing us to isolate problems or
success with the calculation; second, we will test our basic hypothesis, that
magnetic feedback from supernova explosions is an essential piece of the
magnetization of the universe.  Finally, we will make
predictions for measurements that will be made by both current and
future observatories.  These will primarily target next-generation radio
telescopes and their pathfinders, as well as JWST and WFIRST.

Our catalogue of observations will include the following: synchrotron emission
polarization; Faraday rotation measure and rotation measure synthesis; CO
emission, assuming a conversion from CO-to-H$_2$ appropriate for the system
\citet[e.g.][]{Genzel12,Clark15}; star
formation tracers such as H$\alpha$ and FIR luminosity; and Zeeman Splitting in
dense cores.

\vspace{-3mm}
\subsubsection{Extragalactic Observations}
\vspace{-2mm}

We will compare our simulations to Faraday rotation measures from
observations of high-redshift galaxies
\cite{2008Natur.454..302B,2008ApJ...676...70K,1998A&A...329..809A},
the field magnitude and arrangement in local dwarf galaxies,
\cite{2000A&A...355..128C,2011A&A...529A..94C,2012MNRAS.423L.127R,Mao12,2013MNRAS.435..149N,2014A&A...567A.134J},
field strength, structure and coherence in nearby elliptical
\cite{1993A&ARv...4..449W,1996MNRAS.279..229M} and spiral galaxies
\cite{2014arXiv1411.1386V}.  We will also use measurements of
large-scale intergalactic magnetic fields as a constraint
\cite{2010Sci...328...73N}.   
%RM measurements will include also include the
%use of depolarization of background sources to explore the field strength and
%turbulence of galxies
%\citep{Schulman92}.  
Recently techniques have been developed to combine polarized synchrotron
emission with Faraday Rotation depths to probe the full magnetic configuration
\citep{Heald09, Mao15}.
We will primarily aim to reproduce the morphology of observed integrated
polarization angles of galaxies \citep{Stil09}, the alignment of molecular
clouds and spiral arms \citep{2011Natur.479..499L}, and the relation between
field strength and galactic properties such as mass and velocity dispersion
\citep{2014arXiv1411.1386V,Tabatabaei16}.
We will then make predictions for (and thus
motivate) future observations of galactic magnetic field properties that will be
available to the current and future radio telescopes such as the Jansky VLA, ALMA, and LOFAR, and in the
further future the Square Kilometer Array pathfinder telescopes such
as ASKAP, APERTIF, MeerKAT, and
the SKA itself.

An extremely useful probe of the correlation of magnetic field properties and
star formation is the
correlation between the Far Infrared Radiation (FIR) flux and the
Radio Continuum (RC).  A surprisingly tight correlation between these
two fluxes has been studied for several decades \citep{Helou85}.  Both
radiation sources are indirectly related to star formation, and we
will aim to probe this correlation as a function of redshift.  The FIR
comes from re-heating of dust near sites of massive star formation,
while the RC comes form synchrotron radiation of cosmic rays
\citep{Helou93,Niklas97}.  \cite{Murphy06} model the radio profiles
as smoothed version of the FIR profiles, indicating that for the most
part the RC can be seen as a diffusive flux of cosmic rays.  However,
accurate measurement of this relation has not been reproduced in
cosmological simulations.  Modeling the FIR is done in post-production
after the simulations have finished, by measuring the mass
distribution in stars at a given time and using radiative transfer to
measure the optical depth to FIR.  Modeling the RC requires following
the cosmic ray spectrum and the competition between radiation terms.
These include inverse Compton scattering, ionization of neutral gas,
and bremsstrahlung in addition to the synchrotron that is observable in the RC.  \cite{Schleicher13b} propose that at high
redshift, inverse Compton may begin to dominate over synchrotron and the FIR-RC
correlation will break down around $z= 2-4$, depending on the strength
of the field and the star formation rate.  Several of our simulations
will include the cosmic ray treatment incorporated into Enzo and
described in \cite{Salem14,Salem14b}, suitably modified to incorporate
cosmic ray losses by the aforementioned processes.  This will allow us
to predict future observations by ALMA and LOFAR and to explore the
utility of the RC as a probe of highly extincted star formation.

Another important correlation is the ``Star Formation Main Sequence''
(SFMS) of galaxies \cite{Daddi07, Speagle14}. This relation
between star formation rate and stellar mass is a power law that is
essentially constant over time, with a normalization that falls with
time.  This can be interpreted as a relatively quiescent, steady-state
star formation process with cosmic time.  Recent simulation efforts
have reported good agreement with the SFMS at low redshift, but
reduced star formation rate at intermediate redshifts
\citep{Sparre15}.  We will measure this for not only the highly
resolved galaxies, but also the entire population of galaxies formed
from the cosmological simulation.  This will be compared to both the
low redshift behavior \cite{Brinchmann04} and the high redshift
properties \cite{Daddi07}.  This measurement itself is not a probe of magnetic
fields, but as star formation depends on field strength we will
correlate the slope and normalization of this linear relation with
magnetic field strength in the constituent galaxies.

\vspace{-3mm}
\subsubsection{Galactic Observations}
\vspace{-2mm}


One of the primary goals in this work is understanding the
relationship between the large scale magnetic field
of the galaxy and the magnetic fields in molecular clouds.  As discussed in Section \ref{sec:conngalmcs}, there is clear evidence of
coherent magnetic fields in galactic structure, but the importance in
molecular cloud structure and dynamics is ambiguous.  

The primary question is:
Over what scales are the magnetic field aligned, and how does this alignment
imprint on the molecular gas and impact its dynamics?  To properly answer this
question, we will focus on three measurements, ensuring consistency with earlier
observational work and exploring field properties that have not been measured.
These properties are: the properties of individual molecular clouds;
the correlation between field
directions at two points; and correlations between field direction and material
gradients.  

\noindent
{\bf Molecular Cloud Properties.}
The mean magnetic field strength in molecular clouds is challenging to measure,
but potentially of critical importance in star formation theory.  We will reconcile
molecular cloud properties with observational molecular clouds by comparing
synthetic $^{13}\rm{CO}$ maps with linewidth-size and mass-size relations as
measured by the Galactic Ring survey \citep{Jackson06,Roman-Duval10}.  We will
then predict the mean field strength vs. linewidth and size, and discuss the
expected mass-to-flux and kinetic-to-magnetic energy distributions.

\noindent
{\bf Field-Field Correlation.}
We will use 
synthetic polarization and Faraday rotation measure to measure the correlation
length in the galactic scale simulations.  We will also measure the distribution
of alignments of magnetic fields between and within individual molecular clouds,
and between each molecular cloud and the three-dimensional, kpc-scale
mean field.
These will be compared to the well-correlated alignment seen within molecular
clouds \citep{Li09} and  along galactic spiral arms \citep{Fletcher11}.  This will
allow us to ensure the validity of the fields in our galaxies, and predict the
probability of alignment between any pair of molecular clouds.

\noindent
{\bf Field-Mass Correlation.}
We will also explore the relation between field strength and material.  This has
garnered recent attention as magnetic field tends to align with low density HI
as seen in the GALFA-HI survey  \citep{Clark14}, and tends to lie perpendicular
to higher density structure \citep{PlanckXXXII14}.  We will perform synthetic dust
emission and synthetic HI maps to examine these properties.  Dust polarization
results will be compared to the histogram of relative orientation between
polarization angle and column density gradient. For consistent comparison with
the HI measurements, we will perform the Rolling Hough Transform \citep{Clark14} on the
synthetic HI map to determine the linear structure, and correlate that with the
polarization map.  These measurements will assess the validity of the data.  We
will also explore the alignment of clouds themselves on the large scale
patterns, and measure the probability of finding a cloud at a certain distance
along the large scale field.  This will measure the interaction of magnetic
field and structure on the 100-1000 pc scale.

\noindent
{\bf CMB Foregrounds}
We will also produce all-sky maps for the study of CMB foregrounds.  Maps of the
sky are necessary for training image processing and noise reduction algorithms,
an essential part of the success of future missions.  We will provide a suite of
all-sky maps for such purposes.  We will also measure the variation of
$C_\ell^{EE}$  and $C_\ell^{BB}$ with time and magnetic feedback prescription.
This will allow us to understand the constituent physics that sets $\alpha_{EE}$
and $\alpha_{BB}$.



%\red{CMB foreground stuff goes here}



