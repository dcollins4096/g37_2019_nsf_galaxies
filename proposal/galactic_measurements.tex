
One of the primary goals in this work is understanding the
relationship between the large scale magnetic field
of the galaxy and the magnetic fields in molecular clouds.  As discussed in Section \ref{sec:conngalmcs}, there is clear evidence of
coherent magnetic fields in galactic structure, but the importance in
molecular cloud structure and dynamics is ambiguous.  

The primary question is:
Over what scales are the magnetic field aligned, and how does this alignment
imprint on the molecular gas and impact its dynamics?  To properly answer this
question, we will focus on three measurements, ensuring consistency with earlier
observational work and exploring field properties that have not been measured.
These properties are: the properties of individual molecular clouds;
the correlation between field
directions at two points; and correlations between field direction and material
gradients.  

\noindent
{\bf Molecular Cloud Properties.}
The mean magnetic field strength in molecular clouds is challenging to measure,
but potentially of critical importance in star formation theory.  We will reconcile
molecular cloud properties with observational molecular clouds by comparing
synthetic $^{13}\rm{CO}$ maps with linewidth-size and mass-size relations as
measured by the Galactic Ring survey \citep{Jackson06,Roman-Duval10}.  We will
then predict the mean field strength vs. linewidth and size, and discuss the
expected mass-to-flux and kinetic-to-magnetic energy distributions.

\noindent
{\bf Field-Field Correlation.}
We will use 
synthetic polarization and Faraday rotation measure to measure the correlation
length in the galactic scale simulations.  We will also measure the distribution
of alignments of magnetic fields between and within individual molecular clouds,
and between each molecular cloud and the three-dimensional, kpc-scale
mean field.
These will be compared to the well-correlated alignment seen within molecular
clouds \citep{Li09} and  along galactic spiral arms \citep{Fletcher11}.  This will
allow us to ensure the validity of the fields in our galaxies, and predict the
probability of alignment between any pair of molecular clouds.

\noindent
{\bf Field-Mass Correlation.}
We will also explore the relation between field strength and material.  This has
garnered recent attention as magnetic field tends to align with low density HI
as seen in the GALFA-HI survey  \citep{Clark14}, and tends to lie perpendicular
to higher density structure \citep{PlanckXXXII14}.  We will perform synthetic dust
emission and synthetic HI maps to examine these properties.  Dust polarization
results will be compared to the histogram of relative orientation between
polarization angle and column density gradient. For consistent comparison with
the HI measurements, we will perform the Rolling Hough Transform \citep{Clark14} on the
synthetic HI map to determine the linear structure, and correlate that with the
polarization map.  These measurements will assess the validity of the data.  We
will also explore the alignment of clouds themselves on the large scale
patterns, and measure the probability of finding a cloud at a certain distance
along the large scale field.  This will measure the interaction of magnetic
field and structure on the 100-1000 pc scale.




