
%%%%%%%%%%%%%%%%%%%%%%%%%%%%%%%%%%%%%%%%%%%%%%%%%%%%%%%%%%%%%%%%%%%%%%%%%%%%
\vspace{-4mm}
\section{Results from prior NSF Support}
\vspace{-3mm}




% \noindent \textbf{Dr. Brian W. O'Shea:} is currently the PI or co-PI
% of multiple NSF-funded projects.  The grant most relevant to this
% project is an NSF Division of Advanced Cyberinfrastructure Petascale
% Computing Resource Allocation grant (PRAC; ACI 0832662; ``Formation of
% the First Galaxies: predictions for the next generation of
% observatories,'' PI: Brian O'Shea, \$40K, 5/2009-4/2013.
% \textbf{Intellectual merit:} This grant examined high-redshift galaxy
% formation using extremely sophisticated cosmological simulations that
% included both radiation transport and detailed treatments of star
% formation and feedback, and made detailed predictions about the
% transition between metal-free and metal-enriched star formation as
% well as the properties of populations of high redshift galaxies.
% Publications from this grant include
% \cite{2013ApJ...773...83X,2014ApJ...791..110X,2014ApJ...795..144C,2015MNRAS.452.2822S,
% 2015ApJ...807L..12O, 2015ApJ...802....8A}, with additional papers
% currently in the refereeing process.
% \textbf{Broader Impacts:}  This project helped to improve the accuracy
% of simulations of galaxy formation, which is crucial to making
% predictions for a variety of ground- and space-based observational
% campaigns.  It helped to train multiple graduate students and
% postdoctoral researchers, and to make substantial infrastructure improvements  to the
% \enzo\ and \yt\ codebases, which will benefit the user communities of
% those tools.  Also, public talks relating to high redshift galaxies
% and their importance to astronomy and astrophysics were given by
% several of the investigators.


\noindent \textbf{Dr. Brian W. O'Shea:} Dr. O'Shea has recently been
the PI or co-PI of several NSF-funded projects.  The most relevant
recent grant is an ongoing NSF Office of Advanced Cyberinfrsastructure
Petascale Computing Resource Allocation grant (PRAC; ACI 1514580;
``Petascale adaptive mesh simulations of Milky Way-type galaxies and
their environments,'' \$31.5K, 8/1/2015-7/31/2018).  
\textbf{Intellectual merits:} This
project has created an extensive library of simulated
Milky Way-like galaxies, their progenitors, and their environments,
which can be used to explore a wide range of observable astrophysical
phenomena.  The analysis of the simulations is ongoing, and are being
used to interpret recent observations relating to the intergalactic
and circumgalactic medium, galactic and extragalactic magnetic fields,
and high redshift galaxy formation. The simulation data produced
during the course of this project and resulting data products will be
made publicly available, heavily leveraging this investment in
computational resources.  \textbf{Broader impacts:} This project
involves scientists in training at the undergraduate,
graduate, and postdoctoral level.  Datasets have been visualized by
the NCSA, and will be widely disseminated
to the public.  Publications: \citep{Barrow17, Barrow17_FL2,
2017ApJ...847...59H, 2016ApJ...833...84X, 2016ApJ...832L...5X}.

\textbf{Dr. David C. Collins:} Dr. Collins is PI on ``Magnetic Fields in the Formation of
Molecular Clouds, Filaments, and Cores'',  (NSF AST-1616026, \$298,492,
09/01/2016 - 08/31/2019)  which has been
examining collapse of pre-stellar cores and the role of magnetic fields therein,
and is beginning to start zoom-in simulations of molecular cloud collapse.
Data analysis is under way on this project.
\textbf{Intellectual merits:} This project is exploring the early phases of
pre-stellar cores with an aim to find the earliest precursors of star forming
objects.  \textbf{Broader impacts:} This project involves training graduate
students, and the data will be freely available upon the completion of the
project.  Publications: none as this proposal submission.
