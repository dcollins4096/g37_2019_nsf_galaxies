\vspace{-4mm}
\section{Tools}
\label{sec:codes}
\vspace{-3mm}

% \brian  \dave  \red{maybe we want to get rid of this section, lose the yt
% discussion, and move our description of Enzo to section 3.1.3?  (I am
% a big fan of this)}
 
% \subsection{\enzo}
% \label{sec:enzo}
% \vspace{-1mm}

\noindent
{\large \textbf{The \enzo\ code.}}
The calculations described in this proposal will be performed using
the \enzo\ code.  \enzo\ is a publicly available Cartesian adaptive
mesh 
refinement code used for the simulation of cosmological and
astrophysical phenomena

\cite[][\url{http://enzo-project.org}]{2014ApJS..211...19B}.  \enzo\
uses a block-structured adaptive mesh refinement scheme 
\cite{Berger89} to achieve high spatial and temporal resolution, and
combines an N-body adaptive particle-mesh solver for dark matter
dynamics with a Piecewise Parabolic Method (PPM) hydro solver that has
been extensively 
modified for cosmological applications and hypersonic
flows \cite{Bryan95}.  In addition, the code includes a range of other
physics, including both equilibrium and nonequilibrium chemistry and
cooling models, radiative cooling using a Raymond-Smith model and
Cloudy emissivity tables \cite{2008MNRAS.385.1443S}, and prescriptions
for the formation and feedback of both stellar populations and black
holes.  \enzo\ also includes modules for magnetohydrodynamics (both
the Dedner divergence-cleaning method and a constrained transport
method; \cite{Wang:2009a,Wang:2010,Collins10}) and radiation transport
using a ray-casting method and flux-limited diffusion \cite{Wise11,
Reynolds09}.  \enzo\ has been used to model a wide variety of
cosmological and astrophysical phenomena -- most relevant to this
proposal, \enzo\ has been used heavily to study high- and
low-redshift cosmological structure formation \cite{Abel02, Turk09,
2007ApJ...654...66O, 2012MNRAS.427..311W, 2008MNRAS.385.1443S,
2012ApJ...749..140H, 2014ApJ...791...64E, 2013MNRAS.432.1989S}, the
intergalactic and circumgalactic medium
\cite{2007ApJ...671...27H,2011ApJ...731....6S,2013MNRAS.430.1548H,2012ApJ...759..137J},
and the properties of star-forming molecular clouds
\cite{Kritsuk11,Kritsuk11b,Collins11,Collins11b,Collins12,Schmidt13}.

The \enzo\ code has been selected by the NSF to be part of the
Petascale Computing Resource Allocation (PRAC) program under three
separate allocations totaling approximately 300 million core-hours, and was one of
the handful of codes that were first used on Blue Waters when it was
first made available for science runs in early 2012.  Near-perfect
scaling of the code in its unigrid (non-AMR) mode has  been
demonstrated to more than 90,000 CPU cores.  Calculations using AMR do
not scale as well, although reasonable performance at up to several
thousand CPU cores can be expected for large, well-balanced, and
physics-rich AMR runs.  Given that the simulations needed for this
project are well below this scale, \enzo\ will easily handle the
calculations that are necessary for the proposed work.

% \vspace{-3mm}
% \subsection{\yt}
% \label{sec:yt}
% \vspace{-1mm}

\vspace{2mm}
\noindent
{\large \textbf{The \yt\ code.}}
Our simulations will be analyzed and visualized with the \yt\ package
\cite[][\url{http://yt-project.org}]{Turk11}, which is an open source
analysis and visualization toolkit for grid- and particle-based
simulations.  \yt\ was originally developed to work with \enzo, but
now supports numerous simulation codes, including both grid-based and
particle-based astrophysical codes (i.e., FLASH, ART, Gadget,
Gasoline).  Among the capabilities of \yt\ are slices and projections
(both on and off the Cartesian axes); volume renderings; halo finding
and profiling; 1, 2, and 3 dimensional profiles; light-cone
projections; synthetic QSO sight lines (i.e., absorption line spectra);
contouring; and clump-finding.  We will use \yt\ to perform many of
these analysis tasks -- all of which are parallelized -- and will
furthermore use \yt\ to make a variety of synthetic observations of
galaxies using tools that have been developed and used by our
collaboration \citep[e.g.,][]{Barrow17,Barrow17_FL2}.  \yt\ can be 
used as a standalone code and is also
callable from within an \enzo\ simulation, allowing analysis to be
performed on in-memory datasets (a capability that will be very
helpful for analysis of our larger simulations).  \yt\ is highly
parallel, scales well on XSEDE computing resources, and is currently
being optimized for the Stampede2 supercomputer.

