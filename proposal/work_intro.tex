
In this project, we will use high dynamic range magnetohydrodynamical
(MHD) simulations to understand how galactic-scale magnetic fields
affect the formation and evolution of molecular clouds, and to
identify the characteristics of these magnetized clouds that pertain
to star formation.  Specifically, we will use a targeted series of
high resolution MHD simulations of galaxy formation, both in a
cosmological context (Section~\ref{sec:cosmo_sims}) and as isolated, idealized
galaxies (Section~\ref{sec:molcloud_sims}), to explore the evolution
of magnetic fields through cosmic time.  

Each series of simulations will use
three main magnetic evolution assumptions: \emph{dynamo-only}  simulations,
wherein a weak seed field fills space at the beginning of the simulation, and
two \emph{feedback} recipes that vary the amount of field injected through
supernovae.  These \emph{feedback} recipes will use the toroidal injection
implemented in \enzo\ by \citet{Butsky17}.  This model deposits a fraction,
$\sigma$, of supernova energy in a toroidal configuration on the grid.  Our
large-field run will clone \citep{Butsky17} and use
 $10^{43}$erg per $\msun$ of initial
stellar mass, and our low-field run will use $1\%$ of that value.  For each of
the observational diagnostics,
(Section~\ref{sec:observational_comps}) we will perform synthetic observations,
matching noise and telescope resolution of the respective instruments, and
perform Baysian estimation to compare the four populations.  We expect that the
\emph{dynamo-only} simulations will fall short, which will allow us to bracket
the viability of our simulations.  In addition, we will perform the analysis on
the unfiltered simulations in the ``perfect telescope'' limit, which will allow
us to examine the ability of future SKA or Next Generation VLA observations to
explore these regions.

These calculations will be run with the \enzo\ and \enzoe\ codes
(Section~\ref{sec:codes}) using computing time secured through the XSEDE and
PRAC allocation processes (Section~\ref{sec:comptime_sims}).  \enzoe\ is the
developmental exascale successor to \enzo, and will allow us to perform much
larger simulations than \enzo.  The majority of the production simulations in
the proposed work will be suitable for either code.  



% From the cosmological simulation 
% These simulations will be used as initial and boundary conditions 

% \red{This will feed back to cosmological simulations as improved star
% formationa nd feedback algorithms, etc.}

% \red{need to elaborate upon this a bit before we dive into the
%   simulation section: we'll do simulations (described in 3.1), compare
% to specific observations (3.2), and will do this in a way that is
% logically partitioned (3.3).}
