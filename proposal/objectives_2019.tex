\vspace{-6mm}
\section{Objectives and Summary}
\label{sec:objectives}
\vspace{-3mm}

It has been argued, somewhat facetiously, that there is only one magnetic field
line in the universe; ultimately all magnetic environments are connected.
Stellar magnetic fields are intimately tied to their local
interstellar medium (ISM) by winds and supernovae; the field in the ISM is
produced and sustained by the galactic
rotation and turbulence;  the circumgalactic medium (CGM) provides the boundary
conditions for the large scale dynamo, impacting the field in the ISM and
delivering magnetic field to the intergalactic medium (IGM).  
In the proposed
work we will use a combination of cosmological and galactic scale simulations to
probe the following questions:
\begin{enumerate}
    \item How and When was the magnetic field of the Galaxy assembled?
    \item Where is the field distributed within the galaxy, and how does the
        assembly, history, and environment affect the distribution and tangling
        of the magnetic field?
    %This entails the location of the gas within the galaxy,
    %and the correlation of the field with itself (structured vs. unstructured field)
    \item What are the observational consequences of this magnetic field
        distribution? 
\end{enumerate}
While these questions pertain to galaxies of all sizes, in this work we will will
focus on Milky Way sized $(L^*)$ galaxies and the impact on observable quantities.

%and here the field will interact with the
%intergalactic medium of the Local Group. 
We will use two suites of simulations on different time- and size-scales:
one \emph{cosmological}
spanning 25 Mpc to
capture the formation of a Milky Way-sized galaxy cosmological time- and size-scales; and one suite of \emph{isolated galaxy} 
simulations at 100 kpc
that will will simulate the same galaxy with the resolution necessary to capture the turbulence that is
crucial for the dynamics of the ISM.  The cosmological simulations will be used
for the initial and boundary conditions of the isolated galaxy simulations.
Simulations will vary the initial mean magnetic field and the amount of field
ejected from supernovae.
%These
%simulations will contain a rich array of galaxies through cosmic time.  Focus will be
%placed on observable magnetic properties of Milky Way sized galaxies at low
%redshift, in particular synchrotron emission and polarized dust emission in the
%interstellar medium (ISM), which are both caused by structures in the magnetic
%field.  

%Extragalactic and high-redshift observations will be
%used to validate the simulations.
%Secondary questions involving variation in mass and redshift
%will also be explored as 

Simulations will be performed with the open-source code Enzo
\citep{Collins10,2014ApJS..211...19B} which has a long history of
cosmological and galactic MHD simulations.   These simulations will also serve as a development vehicle for 
\enzoe, the exascale successor to Enzo \citep{Bordner12,Bordner18}.  \enzoe, which is under development and
nearly completed, will allow simulations to be performed at a processor count
that is not possible for the current Enzo, 
enabling the dynamic range and resolution necessary to capture the
magnetic field evolutions.  

We will monitor three quantities in our simulations; the mechanism for field
growth, the time scale for growth, and the spatial distribution and correlations
of the magnetic field within and around the galaxy.   

We will also produce several synthetic observations, most notably
synthetic polarized synchrotron and dust emission.
The primary areas of study will be the foregrounds for the polarized cosmic
microwave background (CMB), and the
role of magnetic fields in star forming clouds.
%These will
%be compared to observations to ensure the validity of our simulations, and to
%study the impact of magnetic fields in the ISM in two areas
%The primary measurement will be an analysis of the relative importance of
%several magnetic field production mechanisms; primordial fields frozen in to the
%flow, stellar feedback, and small- and large-scale dynamo.  

We hypothesize that feedback of magnetic fields from
stellar explosions, and subsequent dynamo activity, is responsible for
magnetizing the universe.  We will address these questions within this
framework by:
\begin{enumerate}
\item Performing a suite of galaxy formation simulations from cosmological
initial conditions, using a set of three subgrid models for magnetic field
creation
\item Re-simulate a selection of galaxies at sub-parsec resolution to connect the
magnetic fields to the initial and boundary conditions of their birth.
\item Compare synthetic observations of galaxies at each stage to 
synthetic observations.  With these synthetic observations we will address
our hypothesis, and provide predictions for future observations.

\end{enumerate}
\vspace{-2mm}

Observations show that galaxies at $z
\simeq 2$ (one-third of the present age of the universe) have magnetic
fields that are both organized on large scales and of comparable
magnitude to what we see in the Milky Way today.  Organized, dynamically important magnetic fields are also
ubiquitous in the interstellar medium of present-day galaxies.  The
molecular clouds that are the sites of star formation in nearby
galaxies form out of this magnetized plasma, and measurements show
that these clouds have substantial magnetic fields as well.  These
magnetic fields are possibly quite important for star formation, which
remains one of the most important unsolved problems in astrophysics.
While magnetic fields have been observed in all of these astrophysical
regimes, and there is a clear sequence of events -- galaxies form
molecular clouds, which in turn are the sites of star formation -- the
ways in which magnetic fields tie galaxies to molecular clouds, and
thus potentially affect star formation and the initial stellar mass
function, are poorly understood theoretically.  
There is a
clear need for a detailed, self-consistent model of cosmological
structure formation that can trace the evolution of magnetized gas to
the physical scales relevant for star formation.

